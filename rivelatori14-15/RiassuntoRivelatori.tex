\documentclass[12pt,a4paper,titlepage]{report}
\special{papersize=210mm,297mm}
\usepackage[italian]{babel}
\usepackage{amsmath, amsfonts, amssymb, mathrsfs, dsfont}
\usepackage[utf8]{inputenc}
\usepackage{fancyhdr}
\usepackage{fullpage}
\usepackage{color}
\usepackage{graphicx}
\pagestyle{fancy}
\begin{document}
\begin{titlepage}
\title{Riassunto di rivelatori di particelle}
\author{Simone Bologna\\Universit\`a degli studi di Milano-Bicocca\\Corso di laurea magistrale in Fisica}
\date{}
\end{titlepage}
\pagestyle{plain}
\maketitle
\setcounter{page}{2}
\tableofcontents
\part{Introduzione e manipolazione dei segnali}
\input{./TeX_Files/InterazioneRadiazioneMateria.tex}
\chapter{Cenni di statistica}
La statistica \`e fondamentale per verificare il buon funzionamento della strumentazione ed interpretare il significato delle misure sperimentali.
Si ricorda il concetto di media sperimentale e vera (coincidono solo per N elevati) e di residuo e scarto (il residuo \`e lo scarto dei dati sperimentali
dalla media sperimentale).
Inoltra si ricorda la varianza sperimentale:
\begin{equation*}
s^2 = \frac{1}{N} \sum_{i=1}^N (x_i - \bar{x})^2 = \frac{1}{N-1} \sum_{i=1}^N (x_i - \bar{x}_e)^2
\end{equation*}
dove, non conoscendo la media vera, si usa la sperimentale e si sovrastima.
\section{Distribuzioni}
Spesso si costruiscono modelli statistici per i vari processi e li si confonta con i dati sperimentali.
Modelli ricorrenti sono quello di Bernoulli:
\begin{equation*}
P(x) = \frac{n!}{(n-x)!n!} p^x (1-p)^{(n-x)}
\end{equation*}
con $n$ numero di prove, $x$ numero di prove riuscite e $p$ probabilit\`a di un successo.\\
Un altro modello \`e quello di Poisson, un approssimazione di Bernoulli per basse probabilit\`a:
\begin{equation*}
P(x) = \frac{(np)^{x} e^{-np}}{x!}
\end{equation*}
Questa distribuzione non \`e simmetrica, in quanto il valore pi\`u probabile non coincide con il valore medio $np$.\\
Un ultimo modello \`e quello di Gauss, per basse probabilit\`a e alte medie:
\begin{equation*}
P(x) = \frac{1}{\sqrt{2\pi \bar{x}}}\,\text{exp}\left(-\frac{(x-\bar{x})^2}{2\bar{x}}\right)
\end{equation*}
\section{Confronto con i dati sperimentali}
Misurata sperimentalmente $F(x)$, ovvero la distribuzione sperimentale, si pongono $\bar{x}$ e $\sigma$ nel modo definito prima e si confronta
la distribuzione con $P(x)$. 
Un primo approccio \`e grafico, per conoscere qualitativamente l'andamento, successivamente si procede con il test $\chi^2$.
\begin{equation*}
\chi^2 = \sum \frac{(x-\bar{x})^2}{\bar{x}} = (N-1)\frac{s^2}{\bar{x}}
\end{equation*}
per cui:
\begin{equation*}
\frac{\chi^2}{(N-1)} = \frac{s^2}{\bar{x}}
\end{equation*}
detto anche chi-quadro ridotto.
Se la distribuzione \`e poissoniana, esso deve essere circa 1 in quanto la varianza corrisponde alla media, o comunque tra 0.5 e 1.5; un $\chi^2$ troppo alto vuol dire che la serie di dati non \`e compatibile in quanto le
fluttuazioni sperimentali sono troppo elevate rispetto al modello, nel caso sia piccolo vuol dire che si hanno fluttuazioni troppo piccole.
\section{Singola misura}
Supponendo di poter prendere la una singola misura statistica (ad esempio un conteggio), quello che si pu\`o dire \`e che la gaussiana della distribuzione \`e centrata
su quella misura con varianza pari al valore della misura stessa. 
Ci\`o ci permetter\`a di dire che al 68\% il valore medio sar\`a tra $x-\sqrt{x}$ e $x+\sqrt{x}$.
Inoltre l'errore \% sar\`a:
\begin{equation*}
\sigma_{\%} = \frac{\sigma}{x} = \frac{1}{\sqrt{x}}
\end{equation*}
per cui l'errore diminuisce all'aumentare della statistica.
\section{Ottimizzazione dei conteggi}
Supponiamo di voler eseguire un operazione di misura di rate: abbiamo un rate $S$ dovuto alla sorgente ed un rate $B$ dovuto al fondo.
Noi siamo in grado di misurare il rate del fondo e il rate sorgente+fondo, supponendo di avere un tempo $T$ da suddividere in tempo per la
misura del fondo $T_B$ e tempo per la misura del fondo+sorgente $T_{S+B}$, come devo suddividere il mio tempo per ottenere una misura
pi\`u precisa possibile di $S$?\\
Inanzitutto:
\begin{equation*}
B = \frac{N_B}{T_B}
\end{equation*}
mentre:
\begin{equation*}
S+B = \frac{N_{S+B}}{T_{S+B}}
\end{equation*}
di conseguenza:
\begin{equation*}
S = \frac{N_{S+B}}{T_{S+B}} - \frac{N_B}{T_B}
\end{equation*}
e:
\begin{equation*}
\sigma^2_S = \left(\frac{\sigma_{N_{S+B}}}{T_{S+B}}\right)^2 + \left(\frac{\sigma_{N_B}}{T_B}\right)^2= \frac{N_{S+B}}{T^2_{S+B}} + \frac{N_B}{T_B^2}
= \frac{S+B}{T_{S+B}} + \frac{B}{T_B}
\end{equation*}
A questo punto differenziando la relazione si ha:
\begin{equation*}
2 \sigma_S d\sigma_S = - \frac{S+B}{T^2_{S+B}} dT_{S+B} - \frac{B}{T_B^2} dT_B
\end{equation*}
ponendo $d\sigma = 0$ e tenendo conto che $dT_{S+B} + dT_B=0$ quindi $\frac{dT_{S+B}{dT_{B}}} = -1$ si arriva a dire:
\begin{equation*}
\frac{T_{S+B}}{T_B}  = \sqrt{\frac{S+B}{B}}
\end{equation*}
In particolare si nota che per fondi molto intensi si ha $T_{S+B}=T_B$.
Suddividendo i tempi come trovato, ponendo $\epsilon_{\%}$ l'errore relativo su S si ha che per S intensi rispetto al fondo:
\begin{equation*}
\epsilon_{\%} = \frac{1}{ST}
\end{equation*}
per cui \`e importante avere $S$ elevati, magari migliorando l'efficienza.
Per S deboli e fondi intensi:
\begin{equation*}
\epsilon_{\%} = \frac{4B}{S^2 T}
\end{equation*}
in questo caso \`e importante ridurre il fondo, magari ottimizzando la configurazione di misura.
\section{Limiti di rivelabilit\`a}
Supponiamo di voler rivelare la presenza di una contaminazione dovuta ad un campione in un luogo.
Prima effettueremo una misura di solo fondo, poi una misura in presenza del campione, vogliamo determinare dei livelli che indichino la presenza
di un contaminante riducendo il rischio di falsi negativi e falsi positivi per fluttuazioni statistiche.
Poniamo $N_B$ la misura di tasso effettuata con solo fondo, $N_T$ la misura di tasso effettuata con il campione (quindi fondo+sorgente) e
poniamo $N_S$ il tasso dovuto unicamente alla sorgente.\\
Analizziamo il caso in cui non sia presente contaminazione: chiaramente $N_S = N_T - N_B$ con $\sigma_{S} = \sqrt{\sigma_{T}^2 + \sigma_{B}^2} = \sqrt{N_{T} + N_{B}}$.
Se non \`e presente contaminazione $\bar{N}_B = \bar{N}_T$  e $\sigma_{S}=\sqrt{2} \sqrt{N_B}$; se $N_S<0$ chiaramente non \`e presente
contaminazione, mentre se $N_S>0$ \`e necessario porre un livello di confidenza per determinare la probabilit\`a di avere un falso positivo per fluttuazioni statistiche.
In particolare se si prende un C.L. al 90\%, il 5\% dei casi sar\`a con $N_S$ negativo e il 5\% con $N_S$ positivo, per cui la probabilit\`a
di falsi negativi sar\`a del 5\%.
A un C.L. del 90\% corrisponde $1.645\sigma_S = 2.326\sigma_B$ per discriminare l'assenza di un contaminante, con un limite sui falsi positivi del 5\%.\\
Con $N_S>2.326\sigma_B$ non si pu\`o affermare che sia presente un contaminante:
per esempio se $\bar{N}_S=2.326\sigma_B$ il 50\% dei casi darebbe un falso negativo, per questo vogliamo stabilire una soglia per ridurre al di sotto del 5\% la probabilit\`a di falso negativo.
In presenza di sole fluttuazioni statistiche e $N_S<<N_B$, poniamo $N_D$ il valore minimo di $N_S$ per poter determinare la presenza di un contaminante:
$\sigma_D = \sqrt{2 N_B+N_D} \approx \sqrt{2 N_B}$, in questo caso $N_D$ potr\`a essere minore del C.L. (dando esito negativo) oppure maggiore.
Se si pone $N_D = L_C + 1.645\sigma_D \approx 4.653 \sqrt{N_B}$ avr\`o il 5\% di probabilit\`a di avere falsi negativi, in quanto
il 5\% delle fluttuazioni si pone al di sotto di $L_C$. 
Un calcolo meno approssimato porta a dire:
\begin{equation*}
N_D = 4.653 \sqrt{N_B} + 2.706
\end{equation*}
A questo punto \`e possibile definire un'attivit\`a minima della sorgente per essere individuata:
\begin{equation*}
\alpha = \frac{N_D } {\epsilon_{abs} f T}
\end{equation*}
con $\epsilon_{abs}$ efficienza assoluta del rivelatore, $f$ quantit\`a di radiazione prodotta per disintegrazione e $T$ tempo di misura.
\section{Distribuzione degli intervalli di tempo}
La poissoniana determina le probabilit\`a nei processi con rate:
\begin{equation*}
P(n) = \frac{(rt)^n e^{-rt}}{n!}
\end{equation*}
la distribuzione dei tempi tra 2 eventi successivi si calcola come la probabilit\`a di avere 0 eventi fino a $t$ ed averne uno tra $t$ e $t+dt$:
\begin{equation*}
P(0) rdt= re^{-rt}dt
\end{equation*}
La probabilit\`a che un intervallo $t$ contenga $N$ eventi si calcola come:
\begin{equation*}
P(N-1)rdt=\frac{(rt)^{N-1}e^{-rt}}{(N-1)!}rdt
\end{equation*}
\chapter{Trattamento dei segnali elettrici}
\section{Impedenze}
Tipicamente uno strumento possiede un alta impedenza di ingresso, per non perturbare il segnale e una bassa impedenza in uscita,
per minimizzare la perdita di segnale.\\
L'unica eccezione \`e data dai segnali veloci, dove problemi legati alla riflessione del segnale richiedono l'uso di impedenze adattate,
in questo caso si avr\`a attenuazione del segnale per non avere deformazioni nell'impulso.
\section{Effetto pelle}
L'effetto pelle \`e l'effetto per cui una corrente elettrica alternata tende a scorrere maggiormente lungo la superficie di un conduttore rispetto
alle regioni interne.
Il motivo di questo effetto \`e legato ai campi magnetici interni variabili (per via della corrente alternata) e quindi a correnti
indotte che impediscono il fluire della corrente all'interno del cavo.\\
Il risultato di questo fenomeno \`e un aumento della resistenza del conduttore al crescere della frequenza del segnale elettrico,
in quanto una parte del conduttore non viene utilizzata.
\section{I cavi coassiali}
Sono cavi schermati da una maglia di rame, essa funge da schermo per le interferenze a bassa e altissima frequenza (sopra i 100 kHz).
Non \`e un ottimo schermo per le frequenze intermedie (\textbf{approfondire}): in questo caso si usano cavi a doppia schermatura,
oppure si fanno passare i cavi dentro un tubo di materiale conduttore.\\
La velocit\`a di trasmissione del segnale nel cavo tipica \`e circa il 66\% di $c$, tuttavia esistono speciali cavi ritardanti dove
la velocit\`a pu\`o arrivare al 1\%; la velocit\`a \`e proporzionale a $Z = \sqrt{\frac{L}{C}}$.\\
In un cavo le \textbf{caratteristiche fondamentali} sono l'impedenza, l'induttanza e la capacit\`a per unit\`a di lunghezza.
Se il cavo deve trasportare l'alta tensione, allora \`e importante la massima tensione trasportabile.\\
\subsection{Disturbi nei coassiali}
Ogni cavo \`e soggetto a perdite dissipative legate alla resistenza del conduttore, esse sono trascurabili per lunghezze fino a qualche
decina di metri, salvo per impulsi con tempi di salita molto brevi.
Ad esempio un segnale con fronte di salita di 1 ns subisce evidenti distorsioni anche solo con cavi lunghi 3 m, per via di effetti di
riflessione dell'impulso.\\
La schermatura viene anche utilizzata serve per fare da riferimento di massa comune tra i vari dispositivi e viene connesso allo
chassis del dispositivo, ovvero l'involucro metallico dell'apparecchio.\\
Un effetto che \`e importante evitare \`e il \textbf{ground loop} (\textbf{approfondire}):
se il riferimento di massa fa una forma chiusa, per garantire un potenziale comune pu\`o circolare una corrente continua.
Questa corrente pu\`o subire fluttuazioni che generano correnti spurie nel mio segnale, disturbandolo.
Per evitare questo effetto, \`e importante che ogni strumento abbia un riferimento interno di massa, che l'alimentazione venga fornita
a stella da un unico distributore, che la massa degli strumenti coincida con quella dell'alimentatore e che la corrente venga prelevata
da una sola presa con un circuito dedicato.\\
Segnali di transiente di accensione/spegnimento di un dispositivo possono indurre correnti nella schermatura del coassiale,
provocando disturbi. 
Ad esempio i monitor dei computer introducono importanti disturbi ad alta frequenza.\\
\subsection{Metodi di abbattimento del rumore}
Per abbattere il rumore si pu\`o utilizzare un amplificatore operazionale in configurazione differenziale:
il cavo che trasporta il segnale viene intrecciato con un cavo non connesso al dispositivo, il secondo cavo serve per sondare i disturbi
ambientali a cui \`e soggetto il primo, in questo modo il preamplificatore pu\`o eliminare i disturbi comuni e pulire il segnale in gran parte (\textbf{approfondire}).
\subsection{Impedenza caratteristica e riflessione del segnale}
Nella trasmissione di segnali ci si riferisce a due casi limite:
\begin{itemize}
\item Segnali lenti o a bassa frequenza
\item Segnali veloci o ad alta frequenza
\end{itemize}
Tipicamente in un coassiale la velocit\`a di trasmissione di un segnale e 5 ns/m, se il \textit{risetime} del segnale \`e maggiore del tempo
di salita si parla di segnali \textbf{lenti}, altrimenti parliamo di segnali veloci (ad esempio succede nei segnali provenienti dagli scintillatori plastici.
\textbf{Su cavi lunghi centinaia di metri i segnali provenienti da rivelatori sono veloci.}
\subsubsection{Segnali lenti o a bassa frequenza}
Per i segnali lenti la resistenza serie del cavo \`e trascurabile, purch\'e la lunghezza del cavo non superi le centinaia di metri.\\
La capacit\`a verso massa \`e 50-100 pf/m, questa capacit\`a deve essere la pi\`u piccola possibile, in quanto si somma a quella del rivelatore;
l'ampiezza del segnale misurato \`e inversamente proporzionale a C, per quest\`o \`e importante avere cavi corti.
\subsubsection{Segnali veloci o ad alta frequenza}
\`E fondamentale considerare l'\textbf{impedenza caratteristica}: essa \`e dipendente dal dielettrico e dal diametro del conduttore centrale
e dello schermo, mentre non \`e dipendente dalla lunghezza (essa dipende dal rapporto tra induttanza e capacit\`a per unit\`a di lunghezza del cavo).
L'impedenza caratteristica \`e pari all'impedenza con la quale bisogna chiudere il cavo per poter trasmettere impulsi di tensione senza
avere effetti di riflessione del segnale.\\
Tipicamente se si chiude il cavo con un impedenza infinita si osserva riflessione totale di segnale senza sfasamenti,
mentre se si chiude il cavo con un impedenza nulla si osserva una riflessione totale, ma di verso opposto.
Quando si attacca la strumentazione il cavo vede come resistenza di terminazione la resistenza di ingresso dello strumento.
Tenendo conto che gli strumenti hanno, in generale, una resistenza di ingresso alta si deduce che in presenza di segnali veloci si osservano effetti
di riflessione (l'impedenza tipica di un cavo coassiale \`e 50 $\Omega$.
Per questo motivo quando si lavora in queste condizioni si termina il cavo con una resistenza di shunt di valore pari alla resistenza caratteristica.
Il cavo vede questa resistenza in parallelo con quella di ingresso del dispositivo: essendo quest'ultima molto maggiore dell'impedenza caratteristica
la resistenza vista diventa pari alla resistenza di shunt.\\
\`E importante tenere conto del fatto che spesso i dispositivi studiati per lavorare appositamente con segnali veloci hanno gi\`a una resistenza di ingresso di 50 $\Omega$,
per cui \`e importante verificare questo fatto.
\section{Attenuatori di segnali}
Talvolta i segnali elettrici sono troppo intensi, per questo diventa necessario ricorrere a degli attenuatori di segnale.\\
L'attenuatore pi\`u semplice che si possa pensare \`e un \textbf{partitore di tensione}, tuttavia questa tecnica porta ad avere problemi
di disturbi con segnali ad alta frequenza.\\
Un altro attenuatore utilizzato \`e l'attenuatore a T (figura~\ref{fig:attenuatoreT}), in questo attenuatore la resistenza di uscita $R_0$
pu\`o essere accoppiata con l'impedenza del cavo e ponendo $\alpha=V_i/V_o$  con le resistenze:
\begin{equation*}
R_1 = R_o \frac{\alpha - 1}{\alpha + 1}
\end{equation*}
e
\begin{equation*}
R_2 = R_o \frac{2 \alpha}{\alpha^2 - 1}
\end{equation*}
si ha l'attenuazione desiderata.
\begin{figure}[htbp]
\begin{center}
\includegraphics[scale=1]{./Immagini/AttenutatoreT.png}
\caption{Un attenuatore a T}
\label{fig:attenuatoreT}
\end{center}
\end{figure}
\section{Sdoppiamento del segnale}
Per sdoppiare i segnali lenti \`e possibile usare una T semplice, per i segnali ad alta frequenza \`e necessario prendere qualche accorgimento
aggiuntivo, la figura~\ref{fig:sdoppiatore} mostra come deve essere realizzato lo sdoppiamento:
le resistenze R sono da 16.6 $\Omega$ e vengono poste una per ogni ramo dello sdoppiamento.
Inoltre viene posto uno shunt da 16.6 $\Omega$ in parallelo allo strumento di lettura, 
in questo modo il segnale vede lungo la linea una resistenza da 50 $\Omega$ e non si subiscono riflessioni e disturbi.
Chiaramente il segnale in questo modo viene diviso in due segnali di ampiezza dimezzata rispetto all'originale.
\begin{figure}[htbp]
\begin{center}
	\includegraphics{./Immagini/Sdoppiatore.png}
\caption{Uno sdoppiatore di segnale ad alta frequenza}
\label{fig:sdoppiatore}
\end{center}
\end{figure}
\section{Trasformatore invertente}
Il trasformatore invertente, serve ad invertire la polarit\`a di segnali pi\`u brevi di 100 ns, in alternativa \`e necessario usare altri dispositivi ad-hoc, come gli amplificatori.
\begin{figure}[htbp]
\begin{center}
	\includegraphics[scale=1.00]{./Immagini/TrasformatoreInvertente.png}
\caption{Un trasformatore invertente, il ramo dove \`e presente la messa a terra \`e invertito.}
\end{center}
\end{figure}
\section{La formatura del segnale}
Spesso \`e necessario formare un segnale, ovvero fare in moto che abbia determinati tempi di salita e di discesa dell'impulso,
questo pu\`o essere ottenuto con dispositivi passivi o attivi.
\subsection{Differenziatore CR o filtro passa-alto}
\begin{figure}[htbp]
\begin{center}
\includegraphics[scale=1]{./Immagini/FiltroCR.png}
\caption{Filtro passa-alto}
\label{fig:filtroCR}
\end{center}
\end{figure}
Il filtro passa-alto (fig.~\ref{fig:filtroCR}) pu\`o essere utilizzato per formare il fronte di discesa di un segnale:
all'aumentare della frequenza di taglio ($\propto \tau^{-1}=(RC)^{-1}$) la componente in bassa frequenza del segnale viene smorzata, lasciando solamente le alte frequenze che vanno a 0 pi\`u velocemente.
In conclusione al diminuire di $\tau$ il segnale aumenta la propria velocit\`a di smorzamento.
Questo filtro elimina la componente in bassa frequenza ($\omega \cdot \tau \ll 1$) del segnale, lasciando i segnali sinusoidali ad alta frequenza ($\omega \cdot \tau \gg 1$) inalterati.\\
Un segnale in uscita dal preamplificatore pu\`o essere approssimato con un gradino
\begin{equation*}
V_in(t) = \begin{cases}
0 \text{ se }t<0\\
V_0 \text{ se }t\ge 0
\end{cases}
\end{equation*}
in quanto possiede un fronte di salita molto veloce ed un fronte di discesa molto lento (quasi piatto).
Un segnale di questa forma che entra in un CR esce con questa formatura
\begin{equation*}
V_{out}(t) = V_0 e^{-t/\tau} 
\end{equation*}
ovvero con un fronte di discesa formato. 
Questo perch\`e il fronte di salita essendo veloce (quindi formato quasi unicamente da componenti ad alta frequenza) passa inalterato, mentre il fronte di discesa,
essendo lento (quindi formato quasi unicamente da armoniche a bassa frequenza), viene determinato da quali frequenze basse vengono fatte passare.
\subsection{Integratore RC o filtro passa-basso}
\begin{figure}[htbp]
\begin{center}
\includegraphics[scale=1]{./Immagini/FiltroRC.png}
\caption{Filtro passa-basso}
\label{fig:filtroRC}
\end{center}
\end{figure}
Il filtro passa-basso (fig.~\ref{fig:filtroRC}) pu\`o essere utilizzato per formare il fronte di salita di un segnale:
all'aumentare della frequenza di taglio ($\propto \tau^{-1}=(RC)^{-1}$) vengono ammesse nuove componenti ad alta frequenza che, avendo derivata maggiore, fanno salire il segnale pi\`u rapidamente.
In conclusione al diminuire di $\tau$ il segnale aumenta la propria velocit\`a di smorzamento.
Questo filtro elimina la componente in bassa frequenza ($\omega \cdot \tau \ll 1$) del segnale, lasciando i segnali sinusoidali ad alta frequenza ($\omega \cdot \tau \gg 1$) inalterati.\\
Un segnale in uscita dal preamplificatore pu\`o essere approssimato con un gradino
\begin{equation*}
V_in(t) = \begin{cases}
0 \text{ se }t<0\\
V_0 \text{ se }t\ge 0
\end{cases}
\end{equation*}
Un segnale di questa forma che entra in un RC esce con questa formatura
\begin{equation*}
V_{out}(t) = V_0 (1-e^{-t/\tau})
\end{equation*}
ovvero con un fronte di salita formato. 
Questo perch\`e il fronte di salita essendo veloce (quindi formato quasi unicamente da componenti ad alta frequenza) viene determinato da quali frequenze vengono ammesse,
mentre il fronte di discesa, essendo lento (quindi formato quasi unicamente da armoniche a bassa frequenza), passa inalterato.
\subsection{Formatura CR-RC}
Se io combino i due filtri precedentemente descritti frapponendo fra i due un amplificatore operazionale (fig~\ref{fig:filtroCRRC}) di disaccoppiamento con guadagno pari a 1,
si ottiene una catena di lettura in grado di formare sia il fronte di salita che il fronte di discesa dell'impulso.
\begin{figure}[htbp]
\begin{center}
\includegraphics[scale=1]{./Immagini/FiltroCRRC.png}
\caption{Formatura tramite un CR-RC}
\label{fig:filtroCRRC}
\end{center}
\end{figure}
Se supponiamo di sottoporre la catena ad un gradino di tensione di ampiezza $V_0$ (approssima bene i segnali in uscita da un preamplificatore), in uscita si otterr\`a:
\begin{equation*}
V(t) = V_0 \, e^{-t/\tau_2}(1 - e^{-t/\tau_1}) 
\end{equation*}
Se $\tau_1 \approx \tau_2$ e sviluppando al primo ordine il termine tra parentesi si ottiene:
\begin{equation*}
V(t) = V_0 \, e^{-t / \tau} \frac{t}{\tau}
\end{equation*}
Il tempo caratteristico del RC (passa basso) determina il fronte di salita: diminuendo $\tau$ (=$RC$) aumenta la frequenza di taglio, per cui si allarga la banda di basse frequenze ammesse, aumentando la velocit\`a di salita.
Il tempo caratteristico del CR (passa alto) determina il fronte di discesa: se $\tau$ aumenta, la frequenza di taglio diminuisce, introducendo componenti a bassa frequenza che rallentano la discesa del segnale.
In conclusione aumentare $\tau_{RC}$ aumenta la velocit\`a di salita, aumentare $\tau_{CR}$ diminuisce la velocit\`a di discesa.\\
Le costanti di tempo devono essere scelte in modo tale da poter raccogliere le cariche disponibili, ridurre il rumore elettronico ed evitare il \textit{pile-up}.
In particolare alcune richieste sono in contrapposizione, ad esempio per essere sicuri di raccogliere tutte le cariche pu\`o essere utile avere un tempo di discesa lungo, 
tuttavia questo aumenta il rischio di avere del \textit{pile-up}. 
\begin{figure}[htbp]
\begin{center}
\includegraphics[scale=1]{./Immagini/FormaturaRCCR.png}
\caption{Esempi di segnali formati con varie costanti di tempo}
\label{fig:formaturaRCCR}
\end{center}
\end{figure}
\subsection{Formatura gaussiana}
Costruendo un circuito CR-(RC)$^n$ con n RC in cascata si pu\`o ottenere una formatura gaussiana dell'impulso:
\begin{equation}
V(t) = V_0 \, e^{-t/\tau} (1-e^{-t/\tau})^n \approx V_0 \left(\frac{t}{\tau}\right)^n e^{-t/\tau}
\end{equation}
Questa forma per $n>4$ approssima bene una gaussiana;
il massimo viene raggiunto in $n\tau$, detto anche \textit{\textbf{peaking time}}).
A parit\`a di \textit{peaking time}, questa formatura recupera la linea di base pi\`u velocemente rispetto alla formatura RC-CR.
Questa formatura \`e la migliore in qualit\`a di rapporto segnale-rumore.
\subsection{Formature con fitri attivi}
Utilizzando circuiti con elementi attivi come diodi o transistor si possono ottenere formature pi\`u fantasiose.
\begin{itemize}
\item \textbf{Formatura triangolare}, ottenibile con una serie di filtri attivi
\item \textbf{Formatura trapezoidale}, utilizzata se il risetime \`e variabile, in modo da avere tutta la carica raccolta in rivelatori con grande variabilit\`a di tempi di risposta.
Questa formatura viene ottenuta con circuiti analogici e digitali.
\end{itemize}
\subsection{Formatura CR-RC-CR}
Utilizzata per dare una forma bipolare all'impulso nel caso di rate molto elevati.
\subsection{Formatura con singola linea di ritardo}
La singola linea di ritardo (SDL) viene utilizzata per ridurre la durata di impulsi troppo lunghi:
un segnale viene sdoppiato in due rami, uno \`e il ramo di output, l'altro viene lasciato aperto.
Se il tempo di propagazione $\tau$ in quest'ultimo \`e molto maggiore del tempo di salita dell'impulso, allora dopo $2 \tau$ il segnale ritorna identico sulla linea di output, 
ma invertito.
Sommandosi al segnale precedente lo annulla.
\begin{figure}[htbp]
\begin{center}
	\includegraphics[scale=1]{./Immagini/SDL.png}
	\caption{Formatura con SDL (Single Delay Line)}
\label{fig:SDL}
\end{center}
\end{figure}
Nel caso il segnale avesse un tempo decadimento si pu\`o presentare il problema dell'\textit{undershoot} (tratteggio rosso nell'immagine): per risolverlo
\`e necessario attenuare in modo opportuno il segnale lungo la linea di ritardo.
\subsection{Formatura con doppia linea di ritardo}
\`E possibile rendere il segnale bipolare imponendo un'altra linea di ritardo in uscita dalla SDL con lo stesso tempo della prima linea.
Il problema di questa formatura \`e che non passa da filtri, per cui presenta il problema del rumore non filtrato, per questo viene usata prevalentemente
in rivelatori con poca risoluzione o per i segnali logici.
\section{Cancellazione del polo zero}
Nella realt\`a i nostri dispositivi non sono sottoposti a dei gradini, bensi a segnali che salgono molto velocemente e decadono molto lentamente.
I tempi di decadimento possono portare a degli \textit{undershoot} che vengono recuperati in tempi nell'ordine dei $\mu$s causando problemi nella forma degli
impulsi successivi. \\
Si dimostra che nei CR-RC il problema pu\`o essere risolto utilizzando una resistenza regolabile in parallelo alla capacit\`a nel CR.
\section{Spostamenti della linea di base}
Supponiamo di avere un treno di impulsi, poich\`e in un CR-RC la tensione media deve essere nulla, in caso di alti rate si pu\`o osservare uno spostamento
della linea di base in modo da mantenere tale media nulla.\\
Nel caso di impulsi identici equispaziati lo spostamento non \`e problematico in quanto costante, tuttavia nella realt\`a gli impulsi hanno forma diversa per cui lo spostamento
pu\`o risultare un problema.\\
Per risolvere il problema si pu\`o usare una formatura bipolare in modo da compensare questo effetto, tuttavia porta ad avere alti rapporti rumore-segnale.
Un'altra soluzione proviene dall'accoppiare il segnale in tensione continua che successivamente viene eliminato con un filtro.
\input{./TeX_Files/ImpulsiLineariLogici.tex}
\chapter{Misure temporali}
\section{Trattamento digitale del segnale}
Esistono sistemi spettroscopici digitali, essi si occupano di amplificare e formare il segnale, correggere il polo zero, ristabilire la linea di base,
controllare la stabilit\`a del guadagno e altro.\\
Il punto fondamentale \`e dato dalla velocit\`a di campionamento dell'ADC (Analog to Digital Converter), esso, infatti, campiona e digitalizza i punti con una
certa frequenza; l'inverso di questa frequenza rappresenta la massima precisione temporale che posso raggiungere con il campionamento scelto, 
questa precisione diventa critica quando si trattano segnali veloci, inoltre il campionamento deve essere tale da preservare la forma dell'impulso.\\
Questi problemi devono essere affrontati per ottenere dei grandi vantaggi quali:
\begin{itemize}
\item Una ampia flessibilit\`a nella scelta della formatura con possibilit\`a di utilizzo di formature speciali
\item Un'ottima stabilit\`a del sistema
\item L'eliminazione del rumore dovuto all'elaborazione di segnali lineari
\item Linearit\`a del sistema
\item La possibilit\`a di introdurre ritardi senza distorcere il segnale
\end{itemize}
\section{Il convertitore analogico-digitale}
L'ADC \`e il primo elemento di una catena digitale, esso si occupa di convertire un segnale in una serie di informazioni digitali contenenti il segnale campionato.\\
Nell'ADC sono fondamentali la frequenza di campionamento e la discretizzazione della tensione, infatti a ogni canale corrisponder\`a un particolare $\Delta$V.\\
Si definisce, inoltre, la \textbf{linearit\`a integrale} come la massima deviazione del grafico di conversione tensione - segnale digitale dalla retta,
spesso viene data in percentuale del range totale dell'ADC, ovvero la differenza tra la tensione massima e minima accettabile dal dispositivo.\\
Posto W(k) la larghezza del livello k e Q la larghezza ideale, la non-linearit\`a differenziale pu\`o essere definita come:
\begin{itemize}
\item Il valore massimo al variare di $k$ di:
\begin{equation*}
DNL(k) = \frac{W(k) - Q}{Q}
\end{equation*}
\item La deviazione RMS delle larghezze dei canali da W medio
\item Essa pu\`o anche essere valutata sottoponendo l'ADC ad una rampa crescente e campionandola ad alta frequenza.
Se la DNL \`e nulla allora tutti i canali avranno lo stesso numero di conteggi, altrimenti si osserveranno delle disomogeneit\`a dipendenti linearmente dall'anomala larghezza del canale.
La deviazione massima dalla larghezza $Q$ espressa in unit\`a di bit meno significativi (LSB) ($1Q=1$ bit) viene posta come DNL.
\end{itemize}
\subsection{L'ADC flash}
Il converitore ADC flash \`e formato da una serie di comparatori a soglia crescente; tale soglia viene ottenuta mediante un partitore resistivo,
un registro legge l'ultimo comparatore con segnale non nullo e lo converte nella corrispondente informazione digitale.
Questi dispositivi sono formati da moltissimi comparatori ($2^n$ con $n$ numero di bit), per questo sono soggetti a una DNL scarsa (da 0.5 a 1 LSB) e richiedono un elevata potenza 
(dalle centinaia di mW ai W), tuttavia sono estremamente veloci, dai 100 MHz ai GHz (ovvero meno di un ns per la conversione).
\begin{figure}[htbp]
\begin{center}
\includegraphics[scale=1]{./Immagini/ADCFlash.png}
\caption{Un ADC flash}
\label{fig:ADCFlash}
\end{center}
\end{figure}
\subsection{L'ADC multipasso}
Si pu\`o ottenere un buon compromesso tra velocit\`a e potenza utilizzando un ADC multipasso.
Questi dispositivi utilizzano dei convertitori ADC flash in cascata e si basano su una serie di scale di espansione:
una serie di moduli sincronizzati suddividono in segnale in modo sempre pi\`u fine al proseguire della digitalizzazione, 
ad esempio il primo modulo digitalizza il segnale su 3 bit, il secondo modulo considera la differenza tra il segnale originale e quello digitalizzato e la ridigitalizza
su una scala pi\`u fine e si procede fino ad una digitalizzazione soddisfacente.\\
Per esempio supponendo di avere in ingresso un segnale da 3.5V il primo ADC potrebbe produrre un segnale digitale corrispondente a 3V, il secondo
si occuperebbe di digitalizzare gli 0.5V rimanenti.\\
Dal punto di vista dell'elettronica, il segnale viene ricevuto e sdoppiato, una parte viene ritardata e l'altra viene digitalizzata, un amplificatore
esegue la differenza tra i segnali e lo manda al modulo successivo, lo schema si trova in figura~\ref{fig:ADCMultipasso}.
\begin{figure}[htbp]
\begin{center}
\includegraphics[scale=1]{./Immagini/ADCMultipasso1.png}\\
\includegraphics[scale=1]{./Immagini/ADCMultipasso2.png}
\caption{Schema di ADC Multipasso}
\label{fig:ADCMultipasso}
\end{center}
\end{figure}
Questi dispositivi hanno una velocit\`a nell'ordine dalle decine alle centinaia di MHz, ma richiedono una potenza minore (dalle decine a centinaia di mW)
e hanno DNL inferiori (nell'ordine di 0.5 LSB).
\section{Filtraggio e formatura del segnale digitale}
Quando si effettua un filtraggio di un segnale analogico, il segnale in uscita pu\`o essere visto come un integrale di convoluzione
tra la funzione di risposta del filtro e il segnale in ingresso:
\begin{equation*}
V'(t) = \int_{t-L}^t V(t') H(t-t') dt'
\end{equation*}
con $L$ durata del filtro\footnote{\`E interessante interpretare questo integrale come il prodotto di spettri di Fourier del segnale in ingresso e della funzione di risposta del filtro}.\\
Per un segnale digitale, questo integrale diventa una sommatoria:
\begin{equation*}
V'(j) = \sum_{i=j-L}^j V(i)H(j-i)
\end{equation*}
Supponiamo di voler implementare un filtro digitale, un metodo di calcolo della sommatoria \`e dato dal filtro a scorrimento, figura~\ref{fig:filtroScorrimento}.
In questi filtri, ogni passo di sovrapposizione da un $j$ in sequenza.
\begin{figure}[htbp]
\begin{center}
\includegraphics[scale=0.80]{./Immagini/FiltroScorrimento.png}
\caption{Funzionamento matematico di un filtro a scorrimento trasversale}
\label{fig:filtroScorrimento}
\end{center}
\end{figure}
In questo modo il tempo di filtraggio \`e breve e pu\`o essere effettuato in tempo reale, in quanto il segnale filtrato viene calcolato al volo.\\
Utilizzando questa tecnica sul rumore posso fare dei filtri adattivi per rimuoverlo, allo stesso modo \`e possibile realizzare filtri per pile-up facendoli variare
in base al rate.
\section{Analisi della forma dell'impulso}
La forma dell'impulso \`e in grado di contenere informazioni sul tipo di particella incidente, sull'interazione che ha generato l'impulso e sulla posizione spaziale dell'evento.
Queste tipo di elaborazioni sono molto impegnative computazionalmente, per questo devono essere fatte offline (ovvero in un secondo momento) su segnali digitalizzati e memorizzati.
\subsection{Ristabilimento della linea di base}
Se l'impulso viene digitalizzato \`e possibile campionare la linea di base per poter correggere di volta in volta l'impulso;
questa operazione richiede che per un certo lasso di tempo non avvengano eventi, per cui se si vuole campionare il pi\`u possibile
la linea di base per correzioni, \`e necessario trovare un compromesso con il rate di interazioni.
\subsection{Deconvoluzione di impulsi di pile-up}
Se ho un impulso digtalizzato posso utilizzare un programma di analisi degli impulsi per riconoscere ed eseguire la deconvoluzione degli impulsi di pile-up.
Queste operazioni possono essere fatte interpolando i segnali, in quanto dopo la formatura si ha una espressione analitica per interpolare il segnale.
Chiaramente queste operazioni possono essere fatte unicamente offline e con un campionamento molto spinto.
\section{Estrazione di informazioni temporali}
Ci sono misure in cui \`e importante estrarre il momento di arrivo dell'impulso.
La precisione di questa misura dipende dal rivelatore (raccolta delle cariche libere) e dalla catena elettronica (ad esempio range dinamici ampii possono essere problematici).\\
Le \textbf{unit\`a di trigger} si occupano di produrre un impulso logico ogni qual volta venga discriminato un segnale, questi dispositivi sono soggetti a due problemi (figura~\ref{fig:timeJitter}):
\begin{itemize}
\item \textbf{Time jitter}, ogni segnale ha fluttuazioni casuali nel livello e nella forma dell'impulso dovuti, ad esempio, al rumore oppure a tempi diversi di raccolta delle cariche
che portano segnali legati a eventi identici ad avere istanti di trigger diversi. 
Per ridurre questo problema \`e necessario porre il livello di trigger in regioni dove la pendenza \`e massima, per cui \`e meno sensibile a fluttuazioni.
\item \textbf{Amplitude e rise time walk}, la variabilit\`a nel ampiezza massima del segnale (\textit{amplitude walk}) porta ad istanti diversi di trigger,
lo stesso pu\`o accadere per segnali che variano la loro forma. (\textit{time walk}).
Questi problemi si riducono in portata se si abbassa il livello di trigger.
\end{itemize}
\begin{figure}[htbp]
\begin{center}
\includegraphics[scale=0.70]{./Immagini/TimeJitter.png}
\caption{Time jitter, amplitude walk e rise time walk}
\label{fig:timeJitter}
\end{center}
\end{figure}
\subsection{Leading edge trigger}
Il leading edge trigger fissa un livello di trigger e produce un impulso logico quanto il segnale supera tale livello, questa tecnica di triggering funziona abbastanza bene se i segnali non hanno un'ampiezza troppo variabile.
Il time jitter ci porta ad alzare il livello di trigger, l'amplitude walk ad abbassarlo: \`e presente un ottimo per soglie intorno al 10-20\% del segnale.
\subsection{Trigger sull'istante di crossover dello zero}
Se un segnale \`e bipolare, l'istante di passaggio del livello zero non \`e dipendente dall'ampiezza dell'impulso.
Per questi segnali \`e possibile eseguire un trigger sul momento del crossover, questa tecnica riesce a risolvere bene il problema dell'amplitude walk,
ma amplifica quello del time jitter.\\
Questa tecnica \`e utilizzabile anche sugli scintilllatori, sottoponendo il segnale anodico ad una singola linea di ritardo, purch\`e la forma
degli impulsi non vari molto.
\subsection{Constant fraction timing}
Se il range dinamico dell'impulso \`e ampio, ci si pu\`o svincolare dall'ampiezza dell'impulso eseguendo il trigger su frazioni $f$ dell'ampiezza massima;
in questo modo, a parit\`a di forma viene risolto il problema dell'amplitude walk.\\
Il constant fraction timing viene effettuato in questo modo (fig~\ref{fig:CFT}):
\begin{enumerate}
\item Il segnale viene sdoppiato
\item Una copia viene invertita e ritardata di un tempo maggiore del tempo di salita
\item L'altra copia viene moltiplicata per un fattore $f$ (quindi attenuata)
\item I segnali vengono sovrapposti e il trigger viene posto sull'istante di zero crossover
\end{enumerate}
\begin{figure}[htbp]
\begin{center}
\includegraphics[scale=0.8]{./Immagini/CFT.png}
\caption{Elaborazione di un segnale per il CFT}
\label{fig:CFT}
\end{center}
\end{figure}
Essendo una tecnica basata sul zero crossover, soffre di problemi legati al time jitter, tuttavia risulta un'alternativa migliore in presenza di range dinamici ampii.
\subsection{ARC timing}
L'ARC (Amplitude and Rise time Compensation) timing viene utilizzato in quei rivelatori dove la forma ed il risetime variano, questo accade sopratutto nei rivelatori HPGe.
In questo tipo di timing si suppone che almeno la parte iniziale dell'impulso sia costante e si esegue il trigger sulla parte iniziale dell'impulso.\\
Un sistema ARC esegue queste operazioni(~\ref{fig:ARC}):
\begin{enumerate}
\item Sdoppia il segnale
\item Una copia viene ritardata di un tempo molto inferiore a quello di salita
\item L'altra copia viene invertita ed attenuata
\item I segnali vengono sommati per effettuare un trigger sullo zero crossover
\end{enumerate}
Questo tipo di trigger non compensa l'amplitude walk.
\begin{figure}[htbp]
\begin{center}
\includegraphics[scale=1]{./Immagini/ARC.png}
\caption{ARC timing}
\label{fig:ARC}
\end{center}
\end{figure}
\subsection{ELET timing}
L'ELET (Extrapolated Leading Edge Trigger) si basa sull'estrapolazione del fronte di salita utilizzando una parte che si suppone essere lineare e costante,
anche questa tecnica viene utilizzata su segnali di forma variabile (sempre HPGe).
Una coppia di discriminatori a soglia con rapporto tra le soglie fissato misurano l'intervallo di tempo tra i due punti ed estrapolano all'indietro il punto di inizio
del segnale.
La misura dell'intervallo di tempo viene effettuata con un TAC (Time to Amplitude Converter).
\subsection{FPET timing}
FPET (First PhotoElectron Trigger) esegue il trigger sul primo fotoelettrone in arrivo, pu\`o essere utilizzato negli scintillatori in condizioni di basso rumore sul fotomoltiplicatore.
\subsection{Confronto dei sistemi di timing}
II LET \`e il migliore per segnali con basso range dinamico e forma costante, il CFT \`e migliore per alti range dinamici e forma costante.
ARC e ELET vengono usati prevalentemente sul HPGe.
\section{Spettroscopia temporale}
\subsection{Spettroscopia temporale con un TAC}
Utilizzando un TAC e un MCA \`e possibile eseguire una spettroscopia temporale, in quanto il TAC produce un impulso proporzionale all'intervallo di tempo.\\
La risoluzione temporale del sistema cos\`i realizzato pu\`o essere misurata sdoppiando e ritardando un segnale che successivamente viene posto in ingresso al TAC.
Un sistema ideale mostrera una larghezza di un bin, nella realta si vedr\`a una regione gaussiana, la cui FWHM sar\`a la risoluzione.\\
Supponiamo di avere una sorgente che emette due quanti in coincidenza rivelati da due dispositivi su due rami diversi, eseguendo una spettroscopia temporale ci\`o che si osserver\`a nell'MCA sar\`a dato dalla sovrapposizione
di due grafici (fig~\ref{fig:spettroCoincidenze}):
\begin{itemize}
\item Un picco centrato in $t_f$ (tempo di ritardo di uno dei due rami) la cui area sar\`a il numero di coincidenze prompt. 
La FWHM indicher\`a la risoluzione del sistema. 
Se i due rami della catena elettronica sono simmetrici, allora il picco osservato sar\`a simmetrico, altrimenti si vedranno asimmetrie.
Ad esempio se il secondo ramo (che fa da stop) ha problemi di amplitude walk, si osserver\`a un'asimmetria verso la coda, in quanto l'amplitude walk ritarder\`a l'arrivo del segnale di stop.
\item Un fondo continuo, dovuto a coincidenze casuali. Se i rate di rivelazione dei due rami, $r_1$ e $r_2$, sono molto inferiori rispetto al reciproco della durata di tempo massima
misurabile dal TAC, allora si osserver\`a un fondo costante di concidenze casuali. \\
Per dimostrarlo calcoliamo la probabilit\`a che si verifichi una coincidenza casuale al tempo $T$.
Supponiamo che ci sia stato un evento nel rivelatore 1, la probabilit\`a che avvenga un evento all'istante $T$ \`e data dalla probabilit\`a
che non avvengano eventi fino al temo $T$ per la probabilit\`a che avvenga un evento tra il tempo $T$ e $T+dT$:
\begin{equation*}
P(T) = e^{-T \, r_2} \, r_2 \, dT
\end{equation*}
Moltiplicando per $r_1$ si ottiene il tasso di coincidenze casuali ad un tempo fissato:
\begin{equation*}
r_{12}=r_1\, r_2 \, e^{-T \, r_2} \, dT
\end{equation*}
Se $r_2\ll T^{-1}$ allora $T\,r_2 \ll 1$ e il termine esponenziale pu\`o essere approssimato a 1 ottenendo:
\begin{equation*}
r_{12} = r_1 \, r_2 \, \Delta T
\end{equation*} 
dove si \`e sostituito $dT$ con $\Delta T$ larghezza di un picco del MCA.
\end{itemize}
\begin{figure}[htbp]
\begin{center}
	\includegraphics[scale=1]{./Immagini/SpettroCoincidenze.png}
\caption{Analisi con MCA di coincidenze}
\label{fig:spettroCoincidenze}
\end{center}
\end{figure}
Per migliorare il rapporto tra l'area del picco e del fondo si pu\`o migliorare la risoluzione temporale del TAC e introdurre una selezione delle ampiezze.
Avere un'attivit\`a $n$ minore della sorgente pu\`o aiutare dato che le coincidenze casuali vanno come $n^2$ e le reali come $n$.
\subsection{Spettroscopia temporale con un'unit\`a di coincidenza}
Un'unit\`a di coincidenza produce un impulso logico, ogni qualvolta riceve in ingresso due impulsi entro un tempo $\tau$, esso rappresenta, quindi, una sorta di SCA sul tempo.\\
Questo dispositivo, insieme ad un dispositivo per produrre ritardi variabili, pu\`o essere usato per eseguire spettroscopie temporali.
Riprendendo il sistema a due rivelatori precedente, sottoponiamo un segnale ad un ritardo fissato $t_f$ e l'altro ad un ritardo variabile $t_v$.
Se $\tau = \frac {\Delta T} { 2}$ (l'intervallo registrato \`e $(-\tau , +\tau)$) allora l'unit\`a di coincidenza esegue lo stesso numero di conteggi di un canale dell'MCA:
variando $t_v$, l'unit\`a segnaler\`a le coincidenze temporali con tempo $t_f - t_v$.\\
In questi sistemi \`e importante impostare bene $\tau$, se esso \`e troppo grande esso contegger\`a troppe coincidenze casuali, se \`e troppo piccolo sar\`a difficile vedere
il picco delle coincidenze casuali. 
Il valore migliore risulta $\tau = L/2$ con $L$ larghezza alla base del picco, scegliere dei $\tau$ sopra questa media porter\`a a dei plateau di coincidenze casuali (fig.~\ref{fig:plateau}),
andare sotto ridurr\`a l'altezza del picco, rendendolo pi\`u difficile da individuare.
\begin{figure}[htbp]
\begin{center}
\includegraphics[scale=1]{./Immagini/Plateau.png}
\caption{Esempi di spettri ottenuti con questa tecnica al variare di $\tau$}
\label{fig:plateau}
\end{center}
\end{figure}
Normalmente non si prende il valore minimo di $\tau$, in quanto possono capitare derive temporali che modificano il rate di eventi;
per questo esso viene preso un po' pi\`u grande del minimo.
\subsection{Correzione delle coincidenze casuali}
Nel caso di coincidenze casuali a due la correzione del fondo \`e piuttosto semplice in quanto esso \`e piatto e vale $2 \tau \, r_1 \, r_2$.\\
La situazione diventa pi\`u complicata per le coincidenze multiple, per esempio per le coincidenze a tre esiste un fondo piatto $2 \tau  \, r_1 \, r_2$,
tuttavia esiste anche la possibilit\`a di avere una coincidenza reale a due con una casuale. 
Quest'ultima \`e difficilmente valutabile e spesso si ricorre ad un'analisi sperimentale per determinare il fondo.
\subsection{Determinazione di $\tau$}
Il fondo delle coincidenze a due pu\`o essere utilizzato per determinare $tau$ usando, ad esempio, due sorgenti scorrelate ben schermate oppure utilizzando
una sola sorgente che emetta quanti non in coincidenza.\\
Un'altra possibilit\`a pu\`o venire dal misurare la larghezza del plateau, in questo caso \`e utile una sorgente che emetta un'elevata quantit\`a di 
radiazione in coincidenza.
\subsection{Misura di coincidenze ritardate}
Supponiamo di avere una sorgente che emetta due quanti in sequenza passando per uno stato metastabile a vita media inferiore alla risoluzione temporale del sistema.
In questo caso vedremo un picco gaussiano di coincidenze vere con una coda sulla destra di tipo esponenziale, da essa si pu\`o estrarre le informazioni di interesse sulla vita dello stato.\\
Questa misura pu\`o essere effettuata utilizzando un MCA, oppure utilizzando un'unit\`a di coincidenza che effettui una scansione della regione di interesse.\\
Un'altra di questo tipo viene effettuata con la spettroscopia T.O.F. di neutroni, misurando l'intervallo di tempo tra l'istante di produzione e di arrivo
dei neutroni ad un rivelatore, si pu\`o misurare la loro energia.
\subsection{Misure di attivit\`a}
Supponiamo di avere una sorgente che emette 2 quanti $q_1$ e $q_2$ di radiazione in coincidenza con attivit\`a $S$ senza alcuna correlazione angolare, inoltre immaginiamo di avere un rivelatore
sensibile solo ai quanti di tipo $q_1$ e un altro rivelatore sensibile solo ai quanti di tipo $q_2$.\\
Siano $\epsilon_1$ e $\epsilon_2$ le relative efficienze (comprensive di fattori angolari, di efficienze di interazione e altro), allora i tassi di rivelazione saranno:
\begin{gather}
r_1 = \epsilon_1 \, S\\
r_2 = \epsilon_2 \, S
\end{gather}
mentre il tasso di coincidenze vere rivelate sar\`a:
\begin{equation*}
r_{t} = \epsilon_1 \, \epsilon_2 \, S
\end{equation*}
Chiamando $r_{ch}$ il tasso di coincidenze casuali, allora il tasso di coincidenze totali misurate sar\`a
$r_{12} = r_{t} + r_{ch} = \epsilon_1 \, \epsilon_2 \, S + r_{ch} $, risolvendo il sistema si trova:
\begin{equation*}
S = \frac{r_1 \, r_2}{r_{12}-r_{ch}}
\end{equation*}
ovvero l'attivit\`a della sorgente.\\
Questo questo metodo si pu\`o misurare con una accuratezza del 1\% le attivit\`a delle sorgenti senza avere rivelatori che coprano l'intero angolo solido;
inoltre se un rivelatore copre l'angolo solido, allora si pu\`o rinunciare al bisogno di avere radiazione scorrelata angolarmente.\\
Questo metodo tipicamente viene applicato su coincidenze $\beta-\gamma$.
Un rivelatore sensibile solo ai $\gamma$ pu\`o essere ottenuto interponendo un assorbitore di $\beta$ prima del rivelatore, tuttavia un rivelatore
sensibile ai $\beta$ misura sempre qualche fotone, in questo caso \`e necessario cercare di selezionare le energie.\\
Coincidenze $\gamma - \gamma$ sono ancora pi\`u difficili, in quanto anche selezionando le energie, il fondo per l'effetto Compton da sempre parecchi problemi.
\section{Moduli per misure temporali}
\subsection{Moduli di trigger}
I moduli di trigger andrebbero posti subito dopo il rivelatore, tuttavia questo comporta un peggioramento della FWHM.\\
Per questo se si \`e interessati a mantenere l'informazione sull'energia \`e meglio metterlo dopo lo stadio di preamplificazione, dove,
se la salita dell'impulso \`e veloce, l'informazione temporale viene preservata.\\
L'eccezione viene dagli scintillatori, dove, utilizzando una resistenza di carico da 50 $\Omega$ (che genera segnali pi\`u veloci), il segnale all'anodo pu\`o essere utilizzato per misure temporali, 
mentre utilizzando una resistenza pi\`u grande tra l'ultimo dinodo e l'anodo si pu\`o ottenere un segnale a coda lunga con informazioni sull'energia.\\
Altri moduli di trigger possono richiedere una formatura (ad esempio lo zero crossover), in questo caso vanno posti dopo l'amplificatore,
talvolta i moduli di trigger sono integrati in discriminatori o SCA, in modo da non sacrificare troppo l'informazioni sull'ampiezza.
\subsection{Unit\`a di coincidenza}
Se si \`e interessati ad una misura di sovrapposizione degli impulsi, allora il $\tau$ di coincidenza deve essere pari alla larghezza dell'impulso, mentre
se il circuito \`e sensibile al fronte di salita dell'impulso, allora il $\tau$ pu\`o essere scelto liberamente in base alle necessit\`a.\\
Spesso questi dispositivi possiedono pi\`u ingressi (da 1 a 4) che possono essere attivati o disattivati a seconda delle necessit\`a (concidenze a 2, a 3,...);
uno di questi ingressi a volte funge da anticoincidenza, per inibire il dispositivo quando necessario.
\subsection{TAC}
Questo dispositivo viene utilizzato insieme ad un MCA e produce un segnale lineare proporzionale all'intervallo di tempo tra due segnali di start e stop.
In questi dispositivi \`e fondamentale la linearit\`a, essa pu\`o essere misurata sdoppiando segnali e sottoponendoli a linee di ritardo.\\
Esistono TAC di due tipi:
\begin{itemize}
\item \textbf{A sovrapposizione}, il TAC riceve in ingresso due impulsi logici standard e misura l'area di sovrapposizione dei due segnali di start e stop.
Se essi sono in perfetta coincidenza allora gli impulsi si sovrapporranno perfettamente, altrimenti l'area sar\`a proporzionale all'intervallo di tempo: un integratore misura quest'area e d\`a l'informazione.
Questi dispositivi sono veloci nella misura, ma hanno problemi di linearit\`a e precisione, per cui vengono utilizzati in caso di alti rate.
\item \textbf{A start-stop}, in questi dispositivi i segnali di start e stop iniziano e interrompono l'accumulo di carica a corrente costante su un condensatore.
In questo modo la tensione ai capi sar\`a linearmente proporzionale al tempo, ottenendo un'ottima linearit\`a.
\end{itemize}
\subsection{TDC, Time to Digital Converter}
Non \`e molto sensato dover produrre un impulso lineare per poi digitalizzarlo, per questo esistono dispositivi che producono direttamente impulsi digitali.
In questi moduli viene prodotta un gate della durata dell'intervallo, questo gate controlla l'uscita di impulsi di clock a frequenza costante.\\
La frequenza massima attuale \`e di 1 GHz, per cui si pu\`o avere una precisione massima di 1 ns. 
Questo risulta problematico per impulsi brevi (a 20 ns, l'errore \`e del 5\%), per questo spesso i TDC sono corredati con dilatatori di impulsi
che interpolano e dilatano l'impulso temporale.
\subsection{Sistemi di ritardo}
Utilizzando cavi coassiali di diversa lunghezza \`e possibile realizzare sistemi di ritardo nella scala dei ns, tuttavia sopra i 100 ns questo sistema non \`e utilizzabile.
Per ritardi nei $\mu$s si possono utilizzare cavi coassiali con materiali speciali, tuttavia devono essere usati solo per portare segnali a bassa frequenza,
in quanto i segnali ad alta frequenza vengono fortemente distorti.
Spesso questi sistemi sono integrati in amplificatori per mettere a punto sistemi basati sulla temporizzazione.\\
Per ritardare gli impulsi logici poich\`e non si ha informazioni contenute nella forma si possono usare altri sistemi.
Uno di questi \`e basato sulla discriminazione di una rampa: l'impulso da ritardare avvia la salita di una rampa, quando essa supera un livello di discriminazione
viene prodotto un segnale logico in uscita, variando il livello si possono ottenere i ritardi desiderati.
\subsection{Amplificatori a banda larga e filtri temporali}
Talvolta pu\`o essere fondamentale mantenere l'informazione temporale, in questi casi \`e utile riuscire a mantenere il segnale inalterato nella forma
dopo lo stadio di amplificazione.
Un amplificatore a banda larga esegue questa operazione, amplifica semplicemente il segnale senza effettuare alcuna formatura.
Questo modulo \`e utile ad esempio con l'uscita anodica di un fotomoltiplicatore.\\
Se invece si necessita di una formatura minimale, allora si pu\'o utilizzare un filtro temporale, esso esegue un formatura con tempi
caratteristici inferiori ai comuni amplificatori, mantenendo cos\`i inalterata l'informazione temporale a costo di un rapporto segnale-rumore peggiore.
\section{Pulse Shape Discrimination e Rise Time Discrimination}
La forma dell'impulso \`e in grado di contenere informazioni come il tipo di particella che ha interagito con il rivelatore oppure la raccolta delle cariche.
Queste informazioni possono essere ottenute attraverso la PSD o la RTD.
La prima risulta efficace su segnali lineari veloci (come gli impulsi anodici), la seconda \`e equivalente alla PSD sugli impulsi lenti, come quelli del PRE.\\
La PSD pu\`o essere utile in questi casi:
\begin{itemize}
\item	Discriminazione del fondo $\gamma$ negli scintillatori organici se usati come rivelatori di neutroni veloci
\item Riconoscimento del tipo di particella in alcuni scintillatori inorganici
\item Discriminazione tra particelle a range breve e lungo nei contatori proporzionali (tipo le camere a gas)
\item Eliminazione di impulsi spuri nel Ge e nel Si
\item Reiezione di pile-up, osservando le deformazioni
\end{itemize}
Esistono due approcci possibili, uno \`e basato sul sentire le differenze nel rise time, l'altro \`e basato sull'integrazione del segnale in diversi punti:
\begin{itemize}
\item Con due discriminatori a soglie fissate si possono produrre segnali di start e stop per un TAC e osservare la velocit\`a del fronte di salita
\item Si pu\`o rendere il segnale bipolare e poi utilizzare un crossover. 
L'istante di crossover non dipende dall'ampiezza, bens\`i solo dalla forma, per cui misurando l'intervallo di tempo tra un punto fissato (ad esempio il 10\% del fronte di salita)
e lo zero crossover con un TAC ed eseguendo la spettroscopia con un MCA si pu\`o determinare il tipo di particella (fig~\ref{fig:PSD}). 
Un SCA pu\`o essere utilizzato per selezionare il tipo di particella di interesse.
\item Si pu\`o integrare il segnale in due istanti diversi ed eseguire il rapporto dell'integrazione. Questo rapporto non dipende dall'ampiezza e da un
fattore di discriminazione della forma.
\end{itemize}
Si introduce la figura di merito:
\begin{equation*}
M = \frac{X}{W_a + W_b}
\end{equation*}
questo fattore da una misura della qualit\`a della separazione dei segnali ottenuta con il PSD, esso dipende dal range dinamico dei segnali,
per la definizione di $X$, $W_a$ e $W_b$ vedere figura~\ref{fig:PSD}.
\begin{figure}[htbp]
\begin{center}
\includegraphics[scale=1]{./Immagini/PSD.png}
\caption{PSD con zero crossover e figura di merito}
\label{fig:PSD}
\end{center}
\end{figure}
\input{./TeX_Files/MCA.tex}
\input{./TeX_Files/Pileup.tex}
\part{Tipi di rivelatori}
\input{./TeX_Files/IntroduzioneRivelatoriRadiazione.tex}
\chapter{Scintillatori}
Gli scintillatori si basano sull'emissione luminosa da parte di atomi eccitati:
una particella ionizzante deposita la propria energia sul materiale causando una eccitazione atomica o molecolare e, in seguito,
emissione di radiazione luminosa.
Gli scintillatori hanno usi multipli:
\begin{itemize}
\item Spettroscopia $\gamma$
\item Calorimetria
\item Sistema T.O.F.
\item Sistema di trigger
\item Sistema di veto
\item Traccianti
\end{itemize} 
Possono essere divisi in due macrocategorie:
\begin{itemize}
\item Inorganici, caratterizzati da una buona resa in luce (e quindi una maggiore efficienza), ma anche pi\`u lenti
\item Organici, con una resa in luce minore, ma una maggiore velocit\`a
\end{itemize}
\section{Meccanismo di scintillazione}
\subsection{Scintillatori inorganici}
\begin{figure}[htb]
\begin{center}
\includegraphics{./Immagini/LivelliEnergeticiScintillatore.png}
\caption{Livelli energetici in uno scintillatore inorganico}
\label{fig:livInorganico}
\end{center}
\end{figure}
La figura~\ref{fig:livInorganico} mostra il funzionamento di uno scintillatore inorganico.\\
In uno scintillatore organico per stimolare le attivazioni vengono introdotte delle impurezze, quando un elettrone viene portato
dalla banda di valenza a quella di conduzione entro qualche ns si ricombina con la lacuna emettendo radiazione.
Per via delle impurezze a volte accade che l'elettrone si vada a posizionare in un livello energetico dell'impurezza, questo
livello energetico ''trappola`` pu\`o richiedere tempi lunghi fino alle centinaia di ms per la diseccitazione. 
In questo caso avviene una fosforescenza.\\
Questi tipi di scintillatori hanno un elevato $Z$ e alta densit\`a, rendendoli adatti alla rivelazione di particelle cariche e raggi $\gamma$.
\subsection{Scintillatori organici}
\begin{figure}[htb]
\begin{center}
\includegraphics[scale=1.00]{./Immagini/LivelliEnergeticiScintillatoreOrganico.png}
\caption{Livelli energetici di uno scintillatore organico}
\label{fig:livOrganico}
\end{center}
\end{figure}
Gli scintillatori organici si basano sull'eccitazione di molecole di tipo organico, in particolare queste molecole sono caratterizzate
da orbitali molecolari di tipo $\pi$ tra le molecole di carbonio che possono essere stimolati per emettere luce nell'ultravioletto.\\
Posso avere scintillatori a monocristalli, liquidi o plastici a seconda dell'uso che si intende fare.\\
Il grande pregio di questo tipo di rivelatore sta nella velocit\`a di risposta che \`e nei ns.
Sono, inoltre, molto economici.
\section{Caratteristiche di uno scintillatore ideale}
\begin{enumerate}
\item Alta efficienza di scintillazione
\item Conversione lineare $S = E \cdot L$
\item Trasparenza 
\item Tempo di emissione breve
\item Buone propriet\`a ottiche e meccaniche. Buona maneggiabilit\`a.
\item Indice di rifrazione simile al vetro, per evitare riflessione totale.
\end{enumerate}
\subsection{Efficienza di scintillazione}
Uno scintillazione ideale dovrebbe avere un elevato $S$:
\begin{equation*}
S = \frac{L}{E} 
\end{equation*}
con $L$ energia luminosa e $E$ energia entrante.
Normalmente questo fattore \`e molto contenuto, nell'ordine del 10\%, per il NaI vale 12\% (il migliore).
Il resto dell'energia finisce in fononi e ionizzazione.\\
La \textbf{formula di Birks} (vale sono per gli scintillatori organici) afferma:
\begin{equation*}
\frac{dL}{dx} = \frac{S \frac{dE}{dx}}{1 + k_B \, \frac{dE}{dx}}
\end{equation*}
con $k_B$ costante di proporzionalit\`a di Birks.
Per le particelle $\alpha$ $\frac{dE}{dx}$ \`e molto grande per cui si ricava $\frac{dL}{dx} = \frac{S}{k_B}$, per gli elettroni
$\frac{dE}{dx}$ \`e piccolo per cui il termine al denominatore pu\`o essere approssimato a 1, ottenendo $L = S \cdot E$.\\
L'efficienza dipende dalla temperatura (fig~\ref{fig:traspInorganici}), dalle impurezze (per via del quenching) e si deteriora con il tempo.
\begin{figure}[htbp]
\begin{center}
\includegraphics[scale=1]{./Immagini/TrasparenzaOrganici.png}
\caption{Andamento della trasparenza con la temperatura, esistono temperature ottimali.}
\label{fig:traspInorganici}
\end{center}
\end{figure}
\subsection{Linearit\`a}
La relazione $S = E \cdot L$ deve essere lineare, in particolare $S$ non dovrebbe dipendere dalla posizione dello scintillatore.
In generale $S$ dipende dalla particella, come si \`e visto dalla formula di Birks e per particelle che disperdono molta energia la relazione perde in linearit\`a:
per elettroni sufficientemente energetici (sopra il centinaio di keV) la relazione \`e lineare, per protoni o $\alpha$ non \`e lineare ad alte energie (si osservano
effetti di quenching tra molecole).
\subsection{Trasparenza}
\`E importante che lo spettro di assorbimento si sovrapponga il meno possibile con lo spettro di emissione, in modo da avere un materiale trasparente ai fotoni e
permettere ad essi di uscire dal materiale scintillante.
\subsection{Tempo di emissione}
Per avere un'elevata risoluzione temporale \`e necessario che la costante di tempo $\tau$ sia molto breve:
\begin{equation*}
I(t) = I_0 e^{-\frac{t}{\tau_0}} + I_1 e^{-\frac{t}{\tau_1}} - I_0 e^{-\frac{t}{\tau_p}}
\end{equation*}
con $\tau_0$ tempo di scintillazione, $\tau_1$ tempo di fosforescenza e fluorescenza ritardata e $\tau_p$ tempo per il popolamento dei livelli eccitati.
\subsection{Proprieta ottiche e meccaniche}
Per propriet\`a ottiche si intende che la geometria dello scintillatore deve essere tale da avere una buona raccolta di luce, per questo \`e necessario
che i cristalli abbiano buone propriet\`a meccaniche: essi devono essere, infatti, di dimensioni e forma variabili secondo le necessit\`a.\\
Per quanto riguarda la maneggevolezza alcuni cristalli sono igroscopici, ovvero assorbono l'umidit\`a dell'aria, per questo devono essere isolati e sotto vuoto, e possono
essere fragili.
\section{La raccolta della luce}
Quando uno scintillatore emette luce, essa viene diffusa in modo isotropico, per questo motivo \`e importante avere un buon meccanismo di raccolta della luce.
Per evitare che la luce esca dal cristallo senza andare sul fotocatodo si usa il meccanismo della riflessione totale:
\begin{equation*}
\text{sin} \, \theta_c = \frac{n_1}{n_0}
\end{equation*}
Questo permette di avere minori perdite alla superficie (l'efficienza \`e sul 80\%) tuttavia rende difficoltosa la trasmissione del segnale al fotocatodo.
Per questo sulla superficie rivolta verso il fotocatodo viene posto un materiale di accoppiamento, ovvero un materiale a indice di rifrazione intermedio, come
del grasso ottico o del silicone.
Siccome il fotocatodo lavora con campi elettrici e magnetici intensi \`e necessario porlo ad una certa distanza dal cristallo, pe questo
\`e necessario guidare la luce verso il fotocatodo mediante l'uso di \textbf{guide di luce}, fibre a geometria cilindrica che trasportano la radiazione luminosa.
\section{Il fotocatodo}
Il fotocatodo ha il ruolo di emettere fotoelettroni quando fotoni di sufficiente energia incidono su di esso.
Questo dispositivo non \`e sensibile a tutta la radiazione luminosa: 
per ottenere l'emissione di un fotoelettrone \`e necessario che il fotone incidente
abbia energia sufficiente a portare un elettrone dalla banda di valenza del materiale alla
banda di conduzione ed a farlo uscire dalla superficie del fotocatodo (generalmente $E_{\gamma} > 2$ eV).
Per questo motivo il fotocatodo \`e in grado di rivelare solo una certa porzione dello spettro
luminoso, in genere a partire dal giallo-verde, in base al materiale che lo compone. \\
Per costruire i fotocatodi si utilizzano dei semiconduttori drogati ad affinit\`a elettronica negativa, ad esempio GaP drogato di tipo p con zinco.\\
Quando un elettrone viene portato in banda di conduzione, esso inizia ad eccitare i fononi del cristallo,
disperdendo la propria energia e raggiungendo (generalmente entro 1 ps) il fondo della banda.
Una volta che si trova in questo stato, l'elettrone impiega un tempo nell'ordine dei 100 ps per ricombinarsi con una
lacuna, tornando in banda di valenza.
L'energia sul fondo della banda di conduzione non \`e sufficiente per permettere agli elettroni di fuggire dal fotocatodo:
tra il materiale ed il vuoto esiste, infatti, una barriera di potenziale (detta affinit\`a elettronica) che impedisce agli elettroni di lasciare il semiconduttore.
Gli unici elettroni ad avere energia sufficiente per superare la barriera sono quelli che
hanno eccitato pochi fononi, per cui gli elettroni possiedono un tempo di 1 ps per lasciare il fotocatodo.
Questo tempo \`e molto breve e pone seri limiti allo spessore del materiale, in quanto gli elettroni possono percorrere poco spazio.
Per aumentare l'efficienza dei fotocatodi, essi vengono contruiti in modo da avere un'affinit\`a elettronica negativa e permettere agli elettroni sul fondo della banda di conduzione di avere energia sufficiente per fuggire.
Questo effetto viene ottenuto deponendo uno strato monoatomico di un materiale elettropositivo (ad esempio il cesio) sulla superficie del dispositivo:
poich\'e gli elettroni del cesio sono poco legati, essi vengono attratti dalle lacune del semiconduttore, ionizzando lo strato ed abbassando l'affinit\`a elettronica.
\subsection{Fabbricazione dei fotocatodi}
Esistono 2 categorie di fotocatodi, quelli opachi, con uno spessore maggiore della profondit\`a di fuga degli elettroni e quelli semitrasparenti.
In quelli opachi gli elettroni vengono prelevati dallo stesso lato in cui la luce incide sul materiale, in quelli semitrasparenti avviene l'opposto, in quanto la luce
riesce a raggiungere l'altro estremo del materiale.\\
Ad ogni modo \`e fondamentale che i fotocatodi abbiano uno spessore uniforme, per avere un estrazione uniforme indipendentemente dalla regione colpita dalla radiazione.
Esistono fotocatodi bialcalini o multialcalini a seconda della risposta alla radiazione che si desidera ottenere.
\section{Il fotomoltiplicatore}
\begin{figure}[htbp]
\begin{center}
\includegraphics[scale=0.8]{./Immagini/Fotomoltiplicatore.png}
\caption{Schema di un fotomoltiplicatore}
\label{fig:fotomolt}
\end{center}
\end{figure}
Il rendimento quantico di un fotomoltiplicatore viene definito come:
\begin{equation*}
\text{Q.E.} = \frac{N_{pe}}{N_{fot}} \approx 20\% - 30\%
\end{equation*}
esso dipende dalla lunghezza d'onda del fotone incidente.
Un fotomoltiplicatore si basa sulla moltiplicazione di fotoelettroni attraverso stadi successivi chiamati \textbf{dinodi}.
Quando un fotoelettrone lascia il fotocatodo, esso viene accelerato da un potenziale verso i dinodi, l'urto con i dinodi libera della carica che
attraverso ulteriori urti viene moltiplicata e, infine, raccolta sull'anodo, producendo un segnale elettrico misurabile.
Tipicamente ogni urto con un dinodo porta ad un guadagno da 3 a 50 elettroni, il prodotto di tutti i guadagni da il guadagno globale.
Ad esempio 10 dinodi con guadagno $g=4$ moltiplicheranno di un fattore $G=4^{10}\approx 10^6$.\\
Di particolare importanza \`e il rumore termoionico dovuto ad emissioni termiche di elettroni da parte del fotocatodo:
a 300 K, $k_b T$ vale circa 25 meV, tuttavia per le code della distribuzione si ha che alcuni elettroni hanno un energia maggiore del lavoro di estrazione, 
causando falsi segnali.
Per questo motivo \`e necessario tenere il fotocatodo freddo, in genere per i semiconduttori si hanno dai 100 ai 10000 elettroni/cm$^2$ e s.\\
La \textbf{risoluzione energetica} \`e fortemente dominata dalla moltiplicazione al primo dinodo, dove gli elettroni sono pochi e 
$\frac{\sigma_E}{E} \propto \frac{1}{\sqrt{n}}$.
\subsection{I dinodi}
Supponiamo che un fotone liberi un elettrone e che esso sia accelerato da una differenza di potenziale di 100 V verso il primo dinodo.
Essendo il gap nell'ordine dei 2-3 eV, verranno eccitati circa 30 elettroni, di cui solo 4-5 usciranno dal dinodo per andare verso lo stadio successivo.
Aumentando il potenziale la quantit\`a di elettroni liberati cresce, tuttavia essi verranno liberati pi\`u in profondit\`a.
Per questo motivo esiste un valore ottimale per il potenziale di accelerazione.\\
In genere per i dinodi standard il valore di moltiplicazione \`e 5, per i materiali ad affinit\`a elettronica negativa si possono raggiungere fattori di moltiplicazione
pari a 20-30.
\section{Alimentazione del sistema a dinodi}
Chiaramente \`e necessario che la tensione dell'anodo sia maggiore di quella del fotocatodo, questo pu\`o essere ottenuto in due modi:
\begin{itemize}
\item fotocatodo a massa e anodo a tensione positiva
\item fotocatodo a tensione negativa e anodo a massa
\end{itemize}
Per alimentare i vari stadi di moltiplicazione \`e comodo usare un partitore resistivo (figura~\ref{fig:partitoreDinodi}).
\begin{figure}[htb]
\begin{center}
\includegraphics[scale=0.80]{./Immagini/PartitoreDinodi.png}
\caption{Esempio di partitore resistivo per i dinodi}
\label{fig:partitoreDinodi}
\end{center}
\end{figure}
Chiaramente un sistema di questo tipo porta ad avere delle correnti di fuga che per effetto Joule portano ad un aumento della temperatura del dispositivo.
Questa corrente, tuttavia, non pu\`o essere minimizzata troppo, in quanto deve essere maggiore della corrente di fotomoltiplicazione, in modo da mantenere
i potenziali tra i dinodi costanti.
Questo problema diventa rilevante negli ultimi stadi, dove il numero elevato di elettroni estratti porta a correnti di moltiplicazione intense.\\
Parlando quantitativamente si ha che in un tipico evento di scintillazione vengono liberati 1000 elettroni, un tipico fattore di moltiplicazione in un fototubo
\`e $10^6$, per cui si hanno $10^9$ elettroni nell'ultimo stadio di moltiplicazione.
Una tipica sorgente di laboratorio genera $10^5$ eventi ogni secondo, per un totale di $10^{14}\cdot 1.6 \cdot 10^{-19} = 16 \mu$A di corrente media.
Questa corrente media non provoca grandi riscaldamenti per effetto Joule, per cui una corrente di fuga di questo ordine non \`e tipicamente problematica;
tuttavia, questo ipotizza di avere un flusso continuo di elettroni, mentre nella realt\`a si hanno eventi impulsati di durata molto breve (nell'ordine del ns), per cui
la corrente massima diventa:
\begin{equation*}
i \asymp \frac{10^9 \cdot 10^{-19}}{10^{-9}} \asymp 100 \text{mA}
\end{equation*}
che \`e una quantit\`a notevole.
Per questo motivo ci si accontenta di usare una corrente nell'ordine della corrente media insieme a dei \textbf{condensatori di stabilizzazione}.
I condensatori di stabilizzazione vengono caricati dalla corrente di fuga nei momenti di quiete e servono a fornire la carica nella fase di fotomoltiplicazione.
Se i rate non sono troppo intensi il sistema \`e equivalente.
\section{Caratteristiche dei fototubi}
Le caratteristiche fondamentali sono:
\begin{enumerate}
\item Struttura, ogni fototubo ha una propria forma (circolare, esagonale,...)
\item Propriet\`a temporali, come tempo di risposta e risoluzione
\item Tensione e correnti massime, per determinare il guadagno massimo del dispositivo
\item Sensibilit\`a alla luce e alla potenza radiante (per usi in modo continuo)
\item Corrente di buio, ovvero la corrente che scorre nel dispositivo per catodo non illuminato
\item Linearit\`a del dispositivo, fortemente dipendente dalle fluttuazioni di tensione ai dinodi
\item Impulsi spuri e rumore associati ai fotomoltiplicatori
\item Disuniformit\`a nel fotocatodo, possono essere nella risposta, per via delle fluttuazioni di spessore, o nella raccolta al primo dinodo, 
che determina le fluttuazioni maggiori nella misurazione dell'energia.
\item Variazioni di guadagno in base al tasso di conteggi, per il motivo detto prima della compensazione della carica
\end{enumerate}
\section{Impulsi prodotti dal fototubo}
La legge di produzione dei fotoni \`e $I(t) = I_0$ exp$(-\lambda t)$ per cui la corrente di elettroni sar\`a
\begin{equation*}
 i(t) = i_0 \text{exp}(-\lambda t)
\end{equation*}
Da $\int_0^{\infty} i(t) dt$ si ricava
\begin{equation*}
i_0 = \lambda Q
\end{equation*} 
per cui $i(t) = \lambda Q$ exp$(\lambda t)$.\\
Il circuito di lettura del segnale pu\`o essere visto come un circuito RC, per cui la tensione sar\`a:
\begin{equation*}
V(t) = \frac{1}{\lambda - \theta} \frac{\lambda Q}{C} (e^{-\theta t} - e^{-\lambda t})
\end{equation*}
sovrapposizione degli effetti di resistenza e condensatore.\\
Se il tempo di produzione dei fotoni \`e molto inferiore del tempo di scarica RC del sistema, allora il segnale pu\`o essere approssimato come:
\begin{equation*}
V(t) \approx \frac{Q}{C}  (e^{-\theta t} - e^{-\lambda t})
\end{equation*}
Per cui il tempo caratteristico del fronte di salita sar\`a determinato da $\lambda$ mentre il tempo di discesa da $\theta$, il valore
massimo della tensione sar\`a $\frac{Q}{C}$.
Tipicamente per avere un segnale di questo tipo si utilizzano resistenze grandi e capacit\`a piccole, in modo da avere tempi lunghi di scarica e un 
valore di $\frac{Q}{C}$ grande.\\
Se il tempo di produzione dei fotoni \`e molto grande il segnale pu\`o essere approssimato come:
\begin{equation*}
V(t) \approx \frac{\lambda}{\theta}\frac{Q}{C} (-e^{-\theta t} + e^{-\lambda t}) 
\end{equation*}
Per cui, in modo opposto, i tempi di carica dipendono da RC, quelli di scarica da $\lambda$,
in questo modo dal decadimento del segnale \`e possibile ottenere informazioni sulla durata di produzione dei fotoni, per studi temporali.
Ad ogni modo a $\lambda$ e $\theta$ fissati, $V$ dipende linearmente da $Q$, per cui \`e possibile usare lo scintillatore per fare spettroscopia.
\section{Tempo di risposta e risoluzione temporale}
Un tipico fototubo a 14 dinodi ha una tensione totale di 2 kV, per cui la tensione tra una coppia di dinodi sar\`a di circa 150 V.
Supponendo che dopo un urto con un dinodo un elettrone abbia un energia di 0 eV, l'energia media di un elettrone in un fototubo sar\`a di 75 eV,
a cui corrisponde una velocit\`a di
\begin{equation*}
\beta^2  = 2 \bar{E}_k \approx \left(\frac{1}{60}\right)^2 
\end{equation*}
La lunghezza tipica di un fototubo \`e nella decina di cm, per cui i tempi di transito risultano nei 20 ns.\\
Le principali fluttuazioni di questo tempo sono date dalla distribuzione delle energie di uscita dei fotoelettroni dal fotocatodo (tra i 0 e i 2 eV),
quindi dalle fluttuazioni del tempo di transito dal fotocatodo al primo dinodo: esse sono nel decimo di ns ($0.2 - 0.3$ ns).
\section{Risoluzione energetica dello scintillatore}
Come si \`e detto prima una delle principali debolezze dello scintillatore \`e nella raccolta delle \textbf{cariche al primo dinodo}, dove si ha la maggiore fluttuazione
energetica in base ai fotoelettroni moltiplicati. 
Inoltre incidono sulla risoluzione: il \textbf{rumore elettronico}, le \textbf{disuniformit\`a del fotocatodo}, le \textbf{fluttuazioni nel guadagno del fotomoltiplicatore}
e la \textbf{non-linerarit\`a} della scintillazione.\\
La cariche prodotte dal fototubo sono il punto che incide maggiormente nella risoluzione, proviamo a fare un calcolo:
supponiamo che un fotone da 0.5 MeV incida su uno scintillatore con $S = 12\%$, questo significa che $L = 60$ keV.
Supponiamo che l'energia venga trasportata da fotoni a 3 eV, il numero totale di fotoni sar\`a 20000, per via per effetti di perdit\`a dell'informazione,
al fotocatodo ne arriveranno 15000. 
Un fotocatodo con efficienza quantica del 20\% produrr\`a 3000 elettroni; qui c`\`e la debolezza del sistema, infatti la produzione di elettroni \`e soggetta a questa
fluttuazione:
\begin{equation*}
\frac{\sigma_E}{E}  = \frac{\sigma_N}{N} = \frac{1}{\sqrt{3000}} \approx 1.8\%
\end{equation*}
Ovvero una FWHM di $2.35 \cdot 1.8 = 4.3 \%$.\\
La risoluzione dipende quindi da $\frac{1}{\sqrt{E}}$, in realt\`a sperimentalmente si vede che:
\begin{equation*}
R = \frac{(\alpha + \beta E)^{\frac{1}{2}}}{E}
\end{equation*}
con $\alpha$ e $\beta$ parametri sperimentali.\\
Altre cause di perdit\`a di risoluzione possono venire da \textbf{impurit\`a del cristallo}, da non uniformit\`a nella raccolta e conversione della luce,
nella non-linearit\`a della risposta legata alla distribuzione del numero di elettroni liberati ad energia fissata.
Un altro fattore \`e dato dalla perdit\`a di fotoni lungo il trasferimento verso il fotocatodo.\\
\textbf{Tipicamente la risoluzione viene quotata usando sorgenti a 662 keV o 1333 keV}.
\section{Rumore e impulsi spuri}
Rumore termoionico, fotoni ritardati dal cristallo e fotoni provenienti dai dinodi possono causare l'emissione di impulsi spuri.
Questi impulsi sono facilmente riconoscibili, in quanto dovuti a poche particelle, quindi a bassa energia:
filtrando gli impulsi piccoli posso, quindi, eliminarli.
Per ridurre questo problema posso ridurre la superficie del fotocatodo (riduce l'eff. termoionico), usare correnti di buio pi\`u piccole (per avere temperature minori),
Un'altra soluzione \`e raffreddare il fotocatodo, ma questo comporta la presenza di condensa e un aumento di radiazione uscente dal catodo che, per via del numero
elevato di elettroni, pu\`o comportare un'eccessiva distorsione del campo elettrico.\\
Altre sorgenti di impulsi spuri sono:
\begin{itemize}
\item Radioattivit\`a naturale
\item Radiazioni cosmiche
\item Gas residuo ionizzato, se la presenza diventa eccessiva \`e necessario cambiare il fototubo
\end{itemize}
\chapter{Spettroscopia $\gamma$}
A differenza delle particelle cariche, i fotoni possiedono diversi modi di interagire con la materia, noi non siamo in grado di rivelarli direttamente,
ci\`o che riveliamo \`e l'effetto prodotto sugli elettroni del materiale.
Per questo motivo un buon spettrometro $\gamma$ deve avere un'alta efficienza di conversione fotone-elettrone e deve essere in grado di rivelare efficientemente
tali particelle.
Un buon assorbimento di $e^{\pm}$ si ha con solidi di 1 cm di spessore.\\
Studiamo come reagisce il rivelatore ai tre effetti principali.
\section{Effetto fotoelettrico}
Nel caso di effetto fotoelettrico viene liberato un elettrone di energia $E = E_{\gamma} - E_b$, oltre a raggi X a cascata o elettroni Auger.
Se gli elettroni e i raggi X, tutti questi effetti vengono registrati dal rivelatore che, in conclusione, fornisce l'energia originaria del fotone.
Lo spettro quindi \`e dato da un picco detto anche \textbf{fotopicco}.
\section{Effetto Compton}
\`E il fenomeno prevalente nelle energie comprese tra 100 keV e 5 MeV in base al materiale.
L'energia ceduta agli elettroni dipende dall'angolo di scattering e va da un valore quasi nullo, per angolo nullo a
\begin{equation*}
E_{e}^{MAX} = h \nu \left(\frac{2\alpha}{1 + 2 \alpha} \right)
\end{equation*}
per un angolo di 180 gradi di scattering.\\
Per questo motivo lo spettro prodotto da effetto Compton \`e quello in figura~\ref{fig:spettroCompton} nel caso non venga assorbito il fotone
\begin{figure}[htbp]
\begin{center}
\includegraphics[scale=1]{./Immagini/SpettroCompton.png}
\caption{Spettro prodotto da un effetto Compton}
\label{fig:spettroCompton}
\end{center}
\end{figure}
\section{Produzione di coppie}
\`E il fenomeno dominante per alte energie, la coppia generata rilascia energia cinetica nel materiale e successivamente osserviamo l'annicilazione del positrone con
un elettrone.
Supponendo che tutta l'energia cinetica venga misurata posso osservare 3 picchi diversi a seconda dei fotoni da 511 keV assorbiti (figura~\ref{fig:spettroCoppia}.
\begin{figure}[htbp]
\begin{center}
\includegraphics[scale=1]{./Immagini/SpettroCoppia.png}
\caption{Spettro prodotto dalla produzione di coppie}
\label{fig:spettroCoppia}
\end{center}
\end{figure}
Nel caso entrambi i fotoni da 511 keV vengano misurati allora osserviamo il \textbf{picco di energia piena}, tuttavia pu\`o accadere che uno dei due fotoni (o entrambi)
fuggano dal rivelatore, in questo caso osserviamo il \textbf{picco di fuga singola o doppia}.
Posso anche osservare casi intermedi, come degli effetti Compton e una successiva fuga.
\section{Funzione di risposta degli spettrometri $\gamma$}
\subsection{Caso del rivelatore piccolo}
Un rivelatore piccolo \`e un rivelatore sufficientemente grande da assorbire tutta l'energia cinetica degli elettroni, tuttavia di dimensioni inferiori al libero cammino
medio dei fotoni (che quindi non vengono, per la maggior parte, rivelati).
Lo spettro prodotto da un rivelatore di questo tipo \`e dato dal continuo del Compton, dal picco prodotto dall'effetto fotoelettrico e dal picco di doppia fuga.
\subsection{Caso del rivelatore grande}
Un rivelatore grande dovrebbe avere dimensioni nelle decine di cm, in questo caso tutti gli effetti e i fotoni secondari vengono rivelati, producendo solo 
picchi di energia piena e fotopicchi.
\subsection{Caso del rivelatore intermedio}
\`E il tipico caso di un rivelatore reale.
Posso avere picchi di energia piena insieme fughe di fotoni e scattering Compton multipli.
In questo caso la funzione di risposta del dispositivo viene simulata con metodi MonteCarlo.
Parametri importanti sono il rapporto tra il numero di picchi ad energia piena e il numero di eventi totali, il rapporto tra la fuga doppia e l'energia piena e
quello tra la fuga singola e l'energia piena.\\
La funzione di risposta di questi rivelatori ha delle complicazioni legate a:
\begin{itemize}
\item \textbf{Fuga di elettroni secondari}, che diventano importanti se il fotone \`e molto energetico o se il rivelatore \`e piccolo. Per via delle fughe l'energia misurata
	dal rivelatore sara inferiore rispetto a quella reale, peggiorando il rapporto energia piena/eventi totali.
\item \textbf{Fuga di fotoni da bremsstrahlung}, che diventa importante per elettroni molto energetici. L'energia irradiata in questo modo cresce come $Z^2$, anche in questo
	caso il continuo subisce un abbassamento di energia, oltre ad una distorsione nella forma.
\item \textbf{Fuga di raggi X caratteristici}, questo fenomeno diventa importante per fotoni ad energia minore (per via delle emissioni a cascata del fotoelettrico) e nei rivelatori con superfici
	molto pi\`u grandi rispetto al volume.
	Questo effetto produce dei picchi di fuga evidenti, in quanto i raggi X emessi hanno frequenze ben definite dai livelli energetici atomici.
\item \textbf{Radiazione secondaria prodotta vicino alla sorgente}, come la produzione di fotoni di annichilazione nell'ambiente in seguito ad un decadimento $\beta^+$.
	In questo caso viene osservato un picco a 511 keV (o a 1022 keV se il rivelatore \`e a pozzetto).
	Un altro effetto dovuto all'interazione con l'ambiente \`e la produzione di bremsstrahlung nei decadimenti $\beta^-$: in questo caso il continuo subisce un abbassamento in energia.
\item \textbf{Effetti dovuti a materiali esterni}, la radiazione prodotta dalla sorgente pu\`o interagire con l'ambiente e successivamente entrare nel rivelatore.
	Ad esempio un fotone potrebbe fare dello scattering Compton con l'ambiente e successivamente entrare nel dispositivo, in questo caso vedr\`o un aumento degli eventi nella
	regione a bassa energia. Un'altra possibilit\`a \`e un annichilazione con l'ambiente esterno, oppure la produzione di raggi X nell'ambiente esterno.
\item \textbf{Effetti dovuti alla somma di impulsi}, avviene nel caso di rate troppo intensi o nel caso di coincidenze, in questo caso il rivelatore non risolve
	temporalmente gli eventi e considera come un unico evento degli eventi distinti (\textit{pile-up}).
	In questo caso il rivelatore misurer\`a la somma delle energie e il conteggio degli eventi risulter\`a inevitabilmente falsato.
\end{itemize}
\begin{figure}[htbp]
\begin{center}
	\includegraphics[scale=1]{./Immagini/EffettiGamma.png}
\caption{Possibili interazioni con l'ambiente che disturbano la misura}
\end{center}
\end{figure}
\section{Utilizzo di scintillatori come spettrometri $\gamma$}
La funzione di risposta degli scintillatori dipende dal tipo di cristallo usato, ad esempio il NaI ha un'ottima risoluzione e sezione d'urto
per effetto fotoelettrico, mentre il BGO (Bismuto-Germanio-Ossigeno) ha una sezione d'urto per fotoelettrico migliore (per cui ho meno picchi di fuga), 
ma una risoluzione peggiore in quanto ha un basso fattore di conversione energia-luce.\\
\begin{figure}[htbp]
\begin{center}
\includegraphics[scale=0.7]{./Immagini/ConfrontoRisposta.png}
\caption{Confronto delle funzioni di risposta dei due scintillatori.}
\end{center}
\end{figure}
La risoluzione \`e fondamentale per riconoscere picchi in quanto risoluzioni troppo basse potrebbero nascondere i picchi nel fondo.
\subsection{Risposta ai fotoni ad alta energia}
Quando l'energia dei fotoni aumenta (2-20 MeV), la sezione d'urto della produzione di coppie cresce.
Questo significa che verr\`a prodotta una coppia elettrone-positrone ad alta energia, quindi con una sezione d'urto per
bremsstrahlung maggiore.
Questo implica una maggiore perdit\`a alle superfici, per questo motivo all'aumentare dell'energia il picco di energia piena inizia a diminuire.
Inoltre, aumentando l'energia aumenta il numero di fotoelettroni che vengono prodotti nello scintillatore, poich\`e
FWHM $\propto N$ i picchi di energia piena, fuga singola e fuga doppia si allargano (figura~\ref{fig:confrontoAlteEnergie}).
\begin{figure}[htbp]
\begin{center}
\includegraphics[scale=1]{./Immagini/ConfrontoAlteEnergie.png}
\caption{Spettro prodotto da germanio all'aumentare dell'energia}
\label{fig:confrontoAlteEnergie}
\end{center}
\end{figure}
\subsection{Linearit\`a della misura}
La relazione $L = S \cdot E$ deve essere il pi\`u possibile lineare, in quanto misuro con punti discreti, quindi devo
avere meno errori legati alla discretizzazione di una funzione continua.
La linearit\`a dipende dal tipo di particella e dall'energia: per elettroni la relazione ha una non-linearit\`a piuttosto ridotta;
per i fotoni, poich\`e la sequenza di interazioni che esso fa \`e diversa per ogni fotone che viene rivelato, esiste una maggiore
variabilit\`a, quindi le fluttuazioni sono maggiori, tuttavia comunque mantiene una certa linearit\`a per via del fatto che viene
misurato mediante interazioni con elettroni.
\begin{figure}[htbp]
\begin{center}
\includegraphics[scale=1.00]{./Immagini/LinearitaScintillatore.png}
\caption{Percentuale di deviazione dalla linearit\`a per uno scintillatore. Esiste un range dove essa \`e rispettata.}
\end{center}
\end{figure}
\subsection{Calibrazione in energia dei rivelatori}
Per calibrare i rivelatori si usano delle sorgenti ad energie note, in generale si cerca di usare dei $\gamma$ con energie ben spaziate
su tutto il range energetico di interesse.
I raggi $\gamma$ sono noti con una precisione energetica di $10^{-5}$, per questo \`e importante utilizzare sorgenti note con quella
precisione, per questo motivo esistono delle sorgenti standard di calibrazione:
\begin{itemize}
\item K$_{\alpha}$(W) (tungsteno) a 5.9 keV
\item $^{198}$Au a 412 keV
\item $^{60}$Co a 1333 keV
\end{itemize}
\textbf{Non si possono usare i fotoni di annichilazione} in quanto il decadimento non avviene sempre a riposo.\\
Inoltre se ho una fuga singola o doppia in una regione ben calibrata, posso usarla per controllare la correttezza della calibrazione,
lo stesso vale nel caso dell'emissione in cascata di fotoni.\\
La curva di calibrazione viene ottenuta interpolando i punti con una funzione polinomiale del tipo
\begin{equation*}
E_i = \sum_{n=0}^N a_n C_i^n
\end{equation*}
con $N \approx 4-5$ utilizzando il metodo dei minimi quadrati.\\
Sono state osservate dipendenze dalla direzione di incidenza della radiazione della risposta, per questo motivo \`e necessario calibrare
tenendo conto del lato che verr\`a esposto alla radiazione. 
Queste variazioni sono nell'ordine dei 100 eV, per questo motivo diventano importanti per misure molto precise;
questo accade perch\`e il campo elettrico degli elettroni influisce sulla raccolta dei fotoni.
\subsection{Convenzioni sulle efficienze nella spettroscopia $\gamma$}
Alcuni rapporti sono molto diffusi nella descrizione delle efficienze dei rivelatori:
\begin{itemize}
\item \textbf{Rapporto picco-Compton}, viene definita sul decadimento del $^{137}$Cs  a 662 keV come:
\begin{equation*}
R_{pc} = \frac{\text{Conteggi/canale al canale del massimo}}{\text{Conteggi/canale medi nella regione tra 358 e 382 keV}}
\end{equation*}
oppure sul decadimento del $^{60}$Co a 1333 keV:
\begin{equation*}
R_{pc} = \frac{\text{Conteggi/canale al canale del massimo}}{\text{Conteggi/canale medi nella regione tra 1040 e 1096 keV}}
\end{equation*}
Gli intervalli di energia sono stati scelti in quanto in quella regione non \`e presente radiazione naturale, ma solamente Compton.
Questo parametro serve a dare una misura combinata della FWHM con la fotofrazione $R_{ph}$:
\begin{equation*}
R_{ph} = \frac{\text{Area del picco ad energia piena}}{\text{Area totale dello spettro}}
\end{equation*}
$R_{pc}$ viene peggiorato dallo scattering Compton con l'ambiente.
A parit\`a di fotofrazione si osserva che $R_{pc} \propto \frac{1}{\text{FWHM}}$ e, a parit\`a di FWHM, $R_{pc} \propto R_{ph}$ (\textbf{perch\`e?}).
\item \textbf{Efficienza assoluta del picco ad energia piena} $\epsilon_{ap}$
\item \textbf{Efficienza intrinseca del picco ad energia piena} $\epsilon_{ip}$, tipicamente viene quotato per $^60$Co a 1333 keV
\item \textbf{Volume attivo}, ad alte energie $\epsilon_{ip} \propto V_{att}$
\item \textbf{Efficienza relativa}, a volte l'efficienza viene quotata rispetto all'efficienza di un rivelatore a NaI di $3''\times 3''$ 
con sorgente di $^{60}$Co posta a 25 cm di distanza con fotoni a 1333 keV, questa efficienza vale $\epsilon_{ap} = 1.2 \cdot 10^{-3}$
\item \textbf{Regola del pollice}, poich\`e l'efficienza ha una dipendenza dal volume, a volte viene data per unit\`a di volume, ad esempio per il germanio (rivelatore HPGe) vale:
\begin{equation*}
\epsilon_{rel}^{Ge} \approx \frac{V[\text{cm}^3]}{5}\%
\end{equation*}
Per cui un germanio al 100\% ha un volume di 500 cm$^3$.
\end{itemize}
\input{./TeX_Files/RivelatoriAGas.tex}
\chapter{Rivelatori a semiconduttore}
L'utilizzo di camere a ionizzazione a stato solido con semiconduttori pu\`o semplificare notevolmente la struttura di un rivelatore:
l'elevata densit\`a di portatori di carica permette, infatti, di ottenere rivelatori di dimensioni ridotte (sono sufficienti 300 $\mu$m di Si), inoltre,
grazie alla bassa energia di ionizzazione e alla presenza di portatori di carica positivi e negativi, \`e possibile ottenere segnali sufficientemente
ampii senza ricorrere alla moltiplicazione.
Per utilizzare questi rivelatori \`e necessario che le cariche abbiano una vita media elevata e che la loro mobilit\`a sia sufficiente, pur mantenendo
una corrente di buio ridotta.
Queste caratteristiche sono presenti nel silicio, germanio e GaAs, sebbene i primi due a temperatura ambiente abbiano una corrente di buio troppo elevata.
\section{Semiconduttori intrinseci}
In un semiconduttore intrinseco il numero di portatori di carica in banda di conduzione pu\`o essere determinato attraverso:
\begin{equation*}
n^2 = N_c N_v e^{\frac{E_g}{k_B T}}
\end{equation*}
con $N_c$ numero di stati ai margini della banda di conduzione e $N_v$ in banda di valenza.
Posto $n$ il numero di portatori di carica negativi e $p$ numero di quelli positivi (lacune) vale
che $n=p$ e che $np=n^2$.
In un silicio a temperatura ambiente $n_i \sim 10^{10}$ e in un germanio $\sim 10^{13}$.\\
Per quanto riguarda la resistivit\`a di questi materiali, la velocit\`a delle cariche \`e data dalla relazione:
\begin{equation*}
v = \mu_{e,h} \, E
\end{equation*}
La mobilit\`a delle lacune \`e inferiore, ma dello stesso ordine di grandezza, rispetto a quella degli elettroni;
in generale la resistivit\`a pu\`o essere definita come:
\begin{equation*}
\rho = \frac{1}{q(\mu_n n + \mu_h p)}
\end{equation*}
Nel Si o Ge intrinseco, il numero di portatori di carica liberi per agitazione termica \`e troppo pi\`u grande rispetto a quelli prodotti
dalla ionizzazione, per questo non sono utilizzabili nella forma pura ed \`e necessario ricorrere alle giunzioni di materiali drogati.
\section{Semiconduttori estrinseci o drogati}
Introducendo delle impurezze nel cristallo semiconduttore si effettua l'operazione di drogaggio; mediante il drogaggio del materiale si riescono ad ottenere conducibilit\`a maggiori.
Esistono due tipi di drogaggio:
\begin{itemize}
\item tipo p, si introduce un materiale con un elettrone di valenza in meno e si introduce uno stato intermedio spostato verso la banda di valenza;
in questo modo si favorisce la presenza di lacune nel materiale
\item tipo n, si introduce un materiale con un elettrone di valenza in pi\`u e si introduce uno stato intermedio spostato verso la banda di conduzione;
in questo modo si favorisce l'elettrone eccedente ad entrare nella banda di conduzione
\end{itemize}
Introducendo le impurezze, le energie di ionizzazione vanno nei meV, rendendo il materiale un buon conduttore a temperatura ambiente.\
Chiamando $N_D$ il numero di donori (impurezze tipo n) e $N_A$ il numero di accettori (tipo p) se esse sono numerose si pu\`o affermare
che $n \approx N_D$ (in un tipo n) oppure $p \approx N_A$ (in un tipo p).
Inoltre il numero di portatori di carica di segno opposto potr\`a essere calcolato come:
\begin{equation*}
p \;(n) = \frac{n_i^2}{n \; (p)} \approx \frac{n_i^2}{N_D\;(N_A)}
\end{equation*}
In queste condizioni di drogaggio si parla di \textbf{semiconduttori estrinseci};
se $N_A \asymp N_D$ si parla di materiali compensati, se $N_{A,D} \ge 10^{19} - 10^{20}$ cm$^{-3}$ si parla di materiali pesantemente drogati
in quanto si \`e raggiunto il livello di conducibilit\`a dei metalli.
\section{Rivelatori basati su i semiconduttori}
\`E possibile immaginare un rivelatore che si occupi di raccogliere le coppie elettrone-lacuna prodotte in un cristallo semiconduttore in seguito
alla ionizzazione prodotta da una particella.
In un semiconduttore l'energia media per produrre una coppia lacuna-elettrone \`e nell'ordine dei 3 eV (simile al gap di energia tra le bande),
circa 10 volte inferiore a quello nei gas: questo si ripercuote sulla risoluzione, che migliora notevolmente.
Una stima brutale del miglioramento pu\`o essere fatta in approssimazione di risoluzione poissoniana:
poich\`e, a parit\`a di energia, in un semiconduttore si producono 10 volte pi\`u cariche rispetto ad un gas, allora il rapporto tra le risoluzioni
percentuali ($\propto \frac{1}{\sqrt{N}}$) vale:
\begin{equation*}
\frac{R_{SEMI}}{R_{GAS}} = \frac{\sqrt{N}}{\sqrt{10 N}} \approx \frac{1}{3}
\end{equation*}
per cui la risoluzione migliora di un fattore $\frac{1}{3}$. 
In realt\`a, il miglioramento \`e ancora maggiore per via del fattore di Fano.\\
La configurazione pi\`u semplice attuabile porterebbe a introdurre del silicio puro tra due elettrodi, ma purtroppo l'elevata conducibilit\`a
del silicio porta ad avere correnti di buio troppo grandi rispetto alle correnti prodotte dagli eventi;
per questo motivo \`e necessario ricorrere alle giunzioni tra materiali drogati.
\section{La giunzione p-n}
Consideriamo due semiconduttori drogati uno di tipo p e uno di tipo n.
Da un lato si ha una sovrabbondanza di lacune, dall'altra una sovrabbondanza di elettroni;
quando i due semiconduttori vengono messi a contatto, si forma una corrente detta di diffusione dovuta alla ricombinazione delle lacune con gli elettroni.
Lungo la giunzione si forma un accumulo di carica: lungo il lato p gli elettroni giunti a compensarsi con le lacune causano un accumulo di carica
negativa, dall'altro lato rimane un accumulo di carica positiva.
A questo punto si instaura una seconda corrente, legata alla differenza di potenziale associata alla distribuzione di carica, questa corrente
si compensa con quella di diffusione (sono opposte) dando vita alla situazione di equilibrio.
All'equilibrio (figura~\ref{fig:equilibrioPN}) sono presenti 3 regioni, una di tipo p, una di tipo n ed in mezzo una regione dove sono unicamente presenti i portatori di carica intrinseci.
Quest'ultima regione viene detta di svuotamento.
\begin{figure}[htbp]
\begin{center}
\includegraphics[scale=1]{./Immagini/EquilibrioPN.png}
\caption{Condizione di equilibrio in una giunzione PN}
\label{fig:equilibrioPN}
\end{center}
\end{figure}
Quando viene applicata una differenza di potenziale concorde con quella della regione di svuotamento, la regione si allarga, la corrente di diffusione
diminuisce, mentre quella di generazione (dovuta alla differenza di potenziale alla giunzione) rimane costante.
Come risultato si ottiene un'aumento della corrente di fuga $\propto e^{\frac{qV}{k_B T}}$.
\subsection{Dimensioni della regione di svuotamento}
Supponiamo di aver usato materiali drogati n e p in modo diverso, all'equilibrio la dimensione della regione di svuotamento nelle zone sar\`a asimmetrica.
Se poniamo $N_A$ e $N_D$ la densit\`a lineare di accettori e donori, all'equilibrio vale:
\begin{equation}\label{eq:consCaricaSemicond}
N_A \cdot a = N_D \cdot b
\end{equation}
La densit\`a di carica lineare sar\`a:
\begin{equation*}
\rho (x) = 
\begin{cases}
-e N_A \; \; $se $-a<x<0\\
e N_D \; \; $se $0<x<b
\end{cases}
\end{equation*}
L'equazione di Poisson afferma:
\begin{equation*}
\frac{d^2 \varphi}{dx^2} = \frac{-\rho(x)}{\varepsilon}
\end{equation*}
Integrando una volta la distribuzione di carica si pu\`o ottenere l'opposto del campo elettrico:
\begin{equation*}
-E (x) = 
\begin{cases}
-\frac{e N_A x}{\varepsilon} + c_1 \; \; $se $-a<x<0\\
\frac{e N_D x}{\varepsilon} + c_2 \; \; $se $0<x<b
\end{cases}
\end{equation*}
Imponendo le condizioni al contorno $E(-a)=E(b)=0$ si ottiene:
\begin{equation*}
-E (x) = 
\begin{cases}
-\frac{e N_A (x+a)}{\varepsilon}  \; \; $se $-a<x<0\\
\frac{e N_D (x-b)}{\varepsilon}  \; \; $se $0<x<b
\end{cases}
\end{equation*}
La differenza di potenziale pu\`o essere calcolata tramite:
\begin{equation*}
\Delta V = \left| \int_{-a}^{b} E(x) dx \right| = \frac{e}{2 \varepsilon}(N_A a^2 + N_D b^2)
\end{equation*}
da~\ref{eq:consCaricaSemicond}:
\begin{equation*}
N_D = \frac{N_A \, a}{b}
\end{equation*}
per cui:
\begin{equation*}
\Delta V = \frac{e \, N_A \, a}{2 \varepsilon}(a + b)
\end{equation*}
Se il semiconduttore drogato p debolmente $a \gg b$ e $a \approx d$, con $d$ larghezza della regione di svuotamento, e:
\begin{equation*}
\Delta V = \frac{e \, N_A \, d^2}{2 \varepsilon}
\end{equation*}
da cui si ottiene:
\begin{equation*}
d = \sqrt{\frac{2 \Delta V \, \varepsilon}{e \, N_A}}
\end{equation*}
Essendo:
\begin{equation*}
\rho = \frac{1}{e \mu N}
\end{equation*}
con $N$ numero di portatori di carica maggioritari, si pu\`o anche scrivere:
\begin{equation*}
d = \sqrt{2 \mu \rho \Delta V \, \varepsilon} \approx 0.53 \cdot \sqrt{\Delta V \, \rho}
\end{equation*}
La capacit\`a della giunzione pu\`o essere calcolata come:
\begin{equation*}
C = \frac{\varepsilon}{d} = \sqrt{\frac{e \, N_A \, \varepsilon}{2 \Delta V }}
\end{equation*}
Chiamando $V_b$ il potenziale presente ai capi della regione di svuotamento in assenza di potenziali esterni allora:
\begin{equation*}
\Delta V = V_b + V
\end{equation*}
Aumentando $V$ \`e possibile allargare la regione fino a coprire l'intero semiconduttore; per fare questo, \`e fondamentale unire
un materiale molto drogato con uno poco drogato (altamente intrinseco), in questo modo la regione svuotata risulta priva di portatori
di carica liberi e sensibile alle particelle ionizzanti in quanto all'interno \`e presente un campo in grado di raccogliere le cariche prodotte.
\`E fondamentale un'elevata purezza del materiale, per evitare la presenza di centri di ricombinazione (fermano le lacune) e di intrappolamento (fermano gli elettroni).
\section{Rivelatori al silicio}
\begin{figure}[htbp]
\begin{center}
\includegraphics[scale=0.50]{./Immagini/SchemaRivelatoreSilicio.png}
\caption{Schema di un rivelatore al silicio}
\label{fig:schemaRivela49JG8toreSilicio}
\end{center}
\end{figure}
Il silicio \`e molto utilizzato in quanto \`e disponibile a basso costo; il rivelatore \`e formato da una zona fortemente drogata n, che limita
la regione svuotata e funge da contatto ohmico, e da una zona debolmente drogata p, che serve a formare la giunzione.
Dell'ossido di silicio viene formato sulla superfice per passivare il dispositivo ed infine vengono depositati dei sottili contatti di alluminio
(viene scelto l'alluminio perch\`e \`e un materiale con il quale viene bene).
In genere le dimensioni vengono scelte tra i 300 e i 1000 $\mu$m.\\
La carica indotta sugli elettrodi vale:
\begin{equation*}
q(t) \propto \left(1-\text{exp}\left(-\frac{t}{\varepsilon \rho}\right)\right)
\end{equation*}
Il tempo di raccolta del 85\% delle cariche vale:
\begin{equation*}
\tau_e = 2 \rho \varepsilon = 10 \text{ ns}
\end{equation*}
La temperatura operazionale di questi rivelatori pu\`o essere quella ambiente o anche 77 K (azoto liquido) per avere correnti di buio minori;
esistono pi\`u tipi:
\begin{itemize}
\item A barriera superficiale, usati per la spettroscopia $\alpha$ (spessori nel mm)
\item A diffusione, usati per la spettroscopia $\beta$ (spessori nel mm)
\item A deriva di litio, usati per la spettroscopia di X o $\gamma$ a bassa energia (spessori nella decina di mm)
\item Planari passivati, per misure di posizione nella fisica delle alte energie (spessori nel centinaio di $\mu$m)
\end{itemize}
\subsection{Rivelatori a barriera superficiale}
In un rivelatore a barriera superficiale si parte da un silicio non molto drogato n e vi si deposita chimicamente un materiale con un'elevata densit\`a
di trappole per elettroni (difetti).
Questo materiale si comporta come un drogato p fortemente e genera una regione di svuotamento ampia, pur avendo uno strato morto molto sottile; infine sopra questo materiale viene depositato un elettrodo in oro.
Il pregio principale di questa tecnica \`e l'utilizzo di strati morti sottili che assorbono pochissima energia delle $\alpha$ incidenti,
ma proprio per via di questo strato sono sensibili alla luce ambientale (l'energia dei fotoni visibili \`e superiore rispetto a quella del gap), quindi
devono lavorare in condizioni di buio. 
Spesso per la spettroscopia $\alpha$ si lavora in vuoto, di conseguenza la camera a vuoto ha una luminosit\`a trascurabile.\\
La spettroscopia $\alpha$ ottenuta con questi dispositivi ha risoluzioni energetiche nell'ordine dei 10 keV, con un limite statistico a 3 keV.
\subsection{Rivelatori a diffusione}
Un'altra tecnica per fabbricare questi rivelatori consiste nel prendere un wafer drogato debolmente di tipo p ed esponendo una superficie
a vapori di un drogante di tipo n (ad esempio il fosforo): i vapori si diffondono all'interno del wafer (da 0.1 a 2 $\mu$m) portandolo da un
drogaggio di tipo p a uno di tipo n forte.
A questo punto \`e presente uno strato n e uno p, per cui si forma la regione di svuotamento; il problema principale di questa tecnica risiede
nello spessore eccessivo di strato morto, ci\`o rende il dispositivo poco adatto all'impiego nella spettroscopia $\alpha$, ma buono per l'impiego
in spettroscopia $\beta$.\\
Lo strato morto del dispositivo risulta comunque valutabile studiando l'energia persa in funzione dell'angolo di incidenza.
\section{Rivelatori HPGe}
La bassa energia di gap del germanio rende inadatto l'uso a temperatura ambiente, tuttavia a basse temperature (77 K) esso risulta utilizzabile
per effettuare spettroscopia.
Grazie all'elevata purezza raggiungibile con i cristalli di germanio \`e possibile realizzare rivelatori grandi decine di cm$^3$, inoltre
lo $Z=32$ del Ge per la spettroscopia $\gamma$ ad alta risoluzione.
Vengono ottenute grandi regioni di svuotamento attraverso una giunzione p-i-n.
\begin{figure}[htbp]
\begin{center}
\includegraphics[scale=1]{./Immagini/SchemaHPGe.png}
\caption{Schema di un HPGe, il PRE in questi dispositivi sta nella camera a vuoto, per ridurre al minimo la capacit\`a dei cavi}
\label{fig:schemaHPGe}
\end{center}
\end{figure}
\chapter{Rivelatori per neutroni lenti}
Si parla di neutroni lenti quando si ha a che fare con neutroni con un energia inferiore a 0.5 eV, detta anche \textit{cadmium cutoff}, in quanto
una reazione importante del cadmio si posiziona a quell'energia.
Quando si ha a che fare con i neutroni lenti spesso si eseguono operazioni di rivelazione piuttosto che di spettroscopia, in quanto le reazioni
che permettono di misurare direttamente l'energia del neutrone non hanno una sezione d'urto grande.
Vediamo le reazioni utilizzate.
\section{Reazioni nucleari con neutroni a bassa energia}
Le reazioni che vengono utilizzare nella rivelazione di neutroni sono a sezione d'urto elevata (almeno nel barn) in modo da avere rivelatori
di dimensioni ridotte.
Questo aspetto  \`e importante in quanto vengono utilizzati anche rivelatori a gas.
Sono, inoltre, utilizzati materiali ad elevata abbondanza isotopica o con arricchimento a basso costo, in modo da avere quantit\`a elevate di materiale
a basso costo.\\
Le reazioni che vengono utilizzate coinvolgono spesso fotoni che devo essere in grado di discriminare bene, in modo da distinguerli dal fondo e rivelare
in modo efficace i neutroni.
Sempre per questo scopo, vengono utilizzate quelle reazioni che hanno un Q-valore elevato ($Q=(m_i^{TOT}-m_f^{TOT})c^2$) in modo da avere fotoni ad energia
elevata e distinguerli pi\`u facilmente dal fondo (anche con semplici metodi basati sull'ampiezza); esse devono essere fortemente esotermiche
per svincolarci dall'energia del neutrone incidente (in quanto essa \`e bassa) e non avere energia di attivazione.
I prodotti di reazione sono spesso fotoni, particelle cariche pesanti, pezzi di fissione o altri neutroni, si cerca di utilizzare
quelle reazioni che producono particelle a range piccolo, in modo da avere dimensioni del rivelatore ridotte e misurare completamente l'energia
associata.\\
Se queste richieste sono rispettate, il risultato di una rivelazione di neutroni consiste in un picco posizionato al Q-valore della reazione
(fig.~\ref{fig:piccoNeutroni}).
\begin{figure}[htbp]
\begin{center}
	\includegraphics[scale=1]{./Immagini/PiccoNeutroni.png}
\caption{Picco dovuto alla rivelazione di neutroni lenti}
\label{fig:piccoNeutroni}
\end{center}
\end{figure}
\subsection{La cattura neutronica}
\`E la reazione di maggiore interesse, un neutrone lento, se passa sufficiente tempo vicino al nucleo, viene catturato formando un nuovo nucleo eccitato:
\begin{equation*}
\text{n} + ^A\text{X} \to ^{A+1}\text{X}^*
\end{equation*}
I tempi di vita dello stato eccitato possono essere variabili, se essi sono brevi il decadimento dar\`a luogo ad un nucleo diverso con emissione di particelle
$\alpha$ o protoni, altrimenti si parler\`a di stato metastabile e decader\`a nel ground state con emissione di fotoni.
La sezione d'urto di questo processo dipende fortemente dalla struttura interna del nucleo: esse possono andare da pochi millibarn a migliaia di barn,
la dipendenza della sezione d'urto \`e $\sigma_C \propto \frac{1}{v_n}$, per cui neutroni pi\`u lenti possono essere catturati pi\`u facilmente.\\
Una reazione molto importante \`e quella di formazione del deutone (ovvero il nucleo del deuterio):
\begin{equation*}
\text{n}+\text{p} \to \text{d} + \gamma
\end{equation*}
La sezione d'urto del processo risulta:
\begin{equation*}
\sigma_C = \frac{6.2 \cdot 10^4 \, \text{barn}}{v \, \text{(cm/s)}}
\end{equation*}
\subsection{La fissione}
In seguito alla cattura, un nucleo si scinde in due nuclei di massa comparabile pi\`u qualche neutrone.
La sezione d'urto per questo processo \`e di qualche centinaio di barn per neutroni termici.
\section{Reazioni importanti}
\subsection{Cattura del $^{10}$B}
La reazione \`e
\begin{equation*}
\text{n} + ^{10}\text{B} \to ^7\text{Li} + \alpha
\end{equation*}
Nel 94\% dei casi, il litio prodotto \`e nello stato eccitato e decade in $10^{-13}$ s nel g.s. con emissione di un fotone.
La reazione non dipende dall'energia del neutrone, quindi non pu\`o essere usata per effettuare spettroscopia.
L'energia dei due corpi \`e ben nota: poich\`e \`e noto il Q-valore e il neutrone \`e lento, si pu\`o approssimare tutto come un decadimento back-to-back
ed \`e possibile calcolare l'energia dei singoli corpi (nell'ordine del MeV).
La sezione d'urto del processo \`e circa di 4000 barn per n. termici, poi scala senza strutture con la velocit\`a.
\subsection{La cattura del $^{6}$Li}
\begin{equation*}
\text{n} + ^{6} \text{Li} \to ^3 \text{He} + \alpha
\end{equation*}
La sezione d'urto per n. termici \`e circa 1000 barn, poi scala come l'inverso della velocit\`a fino a 200 KeV dove c'\`e una risonanza.
\subsection{La cattura $^{3}$He}
\begin{equation*}
\text{n} + ^{6}\text{He} \to ^2\text{H} + \text{p}
\end{equation*}
La sezione d'urto \`e circa 5000 barn, poi scala, il rivelatore \`e costoso in quanto $^3$He non \`e molto abbondante, il Q-valore \`e circa 800 keV.
\subsection{Cattura radiativa del $^{157}$Gd}
Il gadolinio \`e un materiale ad altissima sezione d'urto di cattura ($\approx 255000$ barn), questo permette di poter costruire
rivelatori di neutroni molto sottili mantenendo una buona efficienza di rivelazione.
In seguito alla cattura il gadolinio emette fotoni o elettroni di conversione, l'elettrone pi\`u emesso \`e a 72 keV con un branching ratio del 39\%.
Questo elettrone pu\`o essere utilizzato per localizzare l'interazione, utilizzando ad esempio un film fotografico che registri la posizione di interazione.\\
Un'altra tecnica prevede l'uso di scintillatori con una piccola frazione (0.5\%) di Gd per rivelare i neutroni, il problema di questa tecnica sta nel
distinguere l'evento dal fondo.
\section{Rivelatori basati sul boro-10}
\subsection{Tubo proporzionale a BF$_3$}
Un rivelatore possibile \`e basato sulla camera proporzionale riempita con BF$_3$, dove si utilizza boro-10;
il gas opera ad una pressione di 0.5-1 atm e l'abbondanza del boro pu\`o essere portata al 100\% ottenendo ottime efficienze.
Lo spettro atteso dal rivelatore risulta quindi formato da due picchi (fig.~\ref{fig:spettroAttesoBF3}) dovuti ai due possibili decadimenti del boro,
inoltre ci si aspetta che le aree siano in un rapporto di 93:6.
\begin{figure}[htbp]
\begin{center}
\includegraphics[scale=1]{./Immagini/SpettroAttesoBF3.png}
\caption{Spettro atteso in un tubo BF$_3$}
\label{fig:spettroAttesoBF3}
\end{center}
\end{figure}
Sperimentalmente viene osservato uno spettro come in fig.~\ref{fig:spettroSperimentaleBF3}.
\begin{figure}[htbp]
\begin{center}
\includegraphics[scale=1]{./Immagini/SpettroSperimentaleBF3.png}
\caption{Spettro osservato sperimentalmente in un rivelatore BF$_3$}
\label{fig:spettroSperimentaleBF3}
\end{center}
\end{figure}
Il motivo di questo spettro \`e dovuto al fatto che posso avere fughe di radiazione che non vengono misurate (effetto parete), dando luogo
al continuo; in particolare posso avere una fuga di sola $\alpha$ o una fuga di solo litio, ma non di entrambi, in quanto essendo emessi
back-to-back, se una particella viene prodotta vicina alla parete ed esce, l'altra andando in verso opposto verr\`a completamente misurata.\\
I rivelatori sono costruiti con anodi di dimensioni delle decine di $\mu$m e tensioni operative di 2-3 kV;
l'anodo pu\`o essere problematico per la sua capacit\`a di catturare neutroni, per questo vengono utilizzati anodi in alluminio, in quanto hanno
sezioni d'urto piccole.
Un problema in questi rivelatori \`e dato dagli impulsi spuri, essi possono essere generati da shock meccanici o fluttuazioni della corrente di fuga; inoltre hanno il problema dell'invecchiamento legato alla deposizione su anodo e catodo di residui da dissociazione molecolare.\\
Spesso ai neutroni sono associati dei fotoni, la loro discriminazione \`e facile a bassi rate, in quanto essi producono elettroni che liberano poca energia nei gas.
Questo diventa un problema ad alti rate, in quanto effetti di pile-up aumentano l'energia apparente e rendono pi\`u difficile la discriminazione;
l'invecchiamento pu\`o rendere difficile la discriminazione, per questo a volte si procede con una purificazione del tubo, per renderlo pi\`u resistente.\\
Per quanto riguarda l'efficienza, supponendo che il flusso incida contro il tubo in modo assiale, l'efficienza risulta:
\begin{equation*}
\epsilon(E) = 1 - e^{-\Sigma_a(E)\,L}
\end{equation*}
dove $\Sigma_a(E)$ \`e la sezione d'urto macroscopica ad energia E.
Valori tipici sono del 90\% per neutroni termici e del 3.8\% per neutroni a 100 eV.
\subsection{Rivelatori rivestiti con boro}
\`E possibile utilizzare una tradizionale camera proporzionale e successivamente rivestirne le pareti interne con boro per aggiungere la capacit\`a di
rivelazione dei neutroni; con questa tecnica si ottengono rivelatori pi\`u resistenti (il BF$_3$ non \`e il gas ideale per una camera proporzionale) ad alti rate di fotoni.
Il problema di questi rivelatori sta nello spettro: poich\`e i prodotti di reazione sono back-to-back, una delle due particelle non verr\`a rivelata
mentre l'energia della seconda verr\`a misurata in base all'energia che essa ha depositato nello strato morto di boro.
\begin{figure}[htbp]
\begin{center}
\includegraphics[scale=1]{./Immagini/SpettroRivestimentoBoro.png}
\caption{Spettro in un rivelatore con rivestimento al boro.}
\end{center}
\end{figure}
Poich\`e l'energia media depositata \`e inferiore, la discriminazione dei fotoni \`e pi\`u difficile.
\subsection{Scintillatori caricati con boro}
Nei tubi proporzionali c'\`e un grosso spread nel risetime rendendoli inutilizzabili per applicazioni T.O.F., un'alternativa sta nell'utilizzo di
scintillatori sottili caricati con boro.
Il problema di questi dispositivi sta nella difficolta di discriminare il fondo $\gamma$, gli elettroni secondari, infatti, depositano tutta la loro energia
e, inoltre, gli scintillatori organici rispondono con meno luce alle particelle pesanti cariche.
\section{Altri tipi di rivelatori}
\subsection{Rivelatori al litio}
Uno scintillatore tipico \`e il LiI, simile al NaI, la luminosit\`a \`e del 35\% rispetto allo scintillatore al sodio.
I tempi tipici di produzione della luce sono nei 300 ns e per spessori nell'ordine del cm l'efficienza \`e quasi pari a 1, per neutroni lenti.
Lo spettro prodotto \`e un picco di energia piena a 4.78 MeV, corrispondente ad un picco prodotto da elettroni a 4.1 MeV.
Essendo il rendimento in luce da particelle cariche pesanti simile a quello dovuto ad elettroni, la discriminazione del fondo gamma \`e difficile in questi dispositivi;
inoltre essi sono igroscopici.
\subsection{Contatori proporzionali ad elio-3}
I prodotti del decadimento dell'elio-3 hanno un Q-valore non molto grande e un range grande, per questo ci sono problemi legati all'effetto parete che
rendono difficile la discriminazione dai $\gamma$.
Questi dispositivi hanno un counting plateau stretto e una pressione di esercizio maggiore del bar, rendendoli pi\`u efficienti dei rivelatori BF$_3$. 
\input{./TeX_Files/NeutroniVeloci.tex}
\input{./TeX_Files/Bolometri.tex}
\end{document}